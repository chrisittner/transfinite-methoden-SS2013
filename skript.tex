%% -*- coding: utf-8; fill-column: 150 -*-
%\documentclass[headsepline=true]{scrartcl}
\documentclass[headsepline=true,DIV=11]{scrartcl}
\usepackage[utf8]{inputenc}
\usepackage[T1]{fontenc}
\usepackage[ngerman]{babel}
\usepackage[colorlinks=true,linkcolor=black,bookmarks]{hyperref}
\usepackage{amssymb, amsmath, amsthm, enumerate, verbatim, extarrows, marvosym, lmodern, stmaryrd, accents, bbm}

\pagestyle{headings}
\providecommand\thetheorem{}
\renewcommand{\thetheorem}{\arabic{theorem}}
%\newtheorem*{question}[theorem]{Frage}
\newtheorem*{theorem}{Satz}
\newtheorem*{lemma}{Lemma}
\newtheorem*{corollary}{Korollar}
\newtheorem*{remark}{Bemerkung}
\theoremstyle{definition}
\newtheorem*{definition}{Definition}
\newtheorem*{example}{Beispiel}
\newtheorem*{convention}{Konvention}
\newtheorem*{motivation}{Motivation}
\newtheorem*{notation}{Notation}
\newtheorem*{question*}{Frage}
\newenvironment{gelaber}{}{}
\newenvironment{preamble}{}{}
\setlength{\parindent}{0in}
\renewcommand{\bar}[1]{\overline{#1}}
%\renewcommand{\Section}[1]{\section{#1} \setcounter{theorem}{0}\setcounter{remark}{0}}

\newcommand{\Card}{\operatorname{Card}}
\newcommand{\WO}{\operatorname{WO}}
\newcommand{\AC}{\operatorname{AC}}
\newcommand{\ZF}{\operatorname{ZF}}
\newcommand{\ZFC}{\operatorname{ZFC}}
\newcommand{\TI}{\operatorname{TI}}
\newcommand{\TT}{\operatorname{TT}}
\newcommand{\ZL}{\operatorname{ZL}}
\newcommand{\CC}{\operatorname{CC}}
\newcommand{\OI}{\operatorname{OI}}
\newcommand{\Pow}{\operatorname{Pow}}
\newcommand{\EM}{\operatorname{EM}}
\newcommand{\TuL}{\operatorname{TuL}}
\newcommand{\TuI}{\operatorname{TuI}}
\newcommand{\dom}{\operatorname{dom}}
\newcommand{\codom}{\operatorname{codom}}
\newcommand{\nlhd}{\not\!\!\lhd} % Negation von \ldh

\begin{document}
\begin{preamble}
\subject{Vorlesung aus dem Sommersemester 2013}
\title{Transfinite Beweismethoden}
\author{Dr. Schuster}
\date{}
%\publishers{\small ge\TeX{}t von }

\maketitle
%\thispagestyle{empty}
%\tableofcontents
%\clearpage
\end{preamble}

\setcounter{section}{0}

\marginpar{\tiny{6.06.2013}}

\subsection*{References}

\begin{gelaber}
	\begin{enumerate}
		\item A. Kertesz. {\em Einführung in die Transfinite Algebra} (main reference)
		\item I. Kaplanski. {\em Set Theory and Metric Spaces}, AMS 2001
		\item G. H. Moore. {\em Zermelo's Axiom of Choice}, Springer
		\item T. Jech. {\em The Axiom of Choice}, North-Holland
		\item H. Rubin, J. E. Rubin. {\em Equivalents of the Axiom of Choice},  Vol. I + II
		\item M. Erne. {\em Einführung in die Ordnungstheorie}, 1982
		\item H. Herrlich. {\em Axiom of Choice}, Springer, 2006
		\item P. Howard, J.E. Rubin. {\em Consequences of the Axiom of Choice}, AMS, 1998
		\item J.L. Bell. {\em The Axiom of Choice}, College, 2009
                \item G. Birkhoff. {\em Lattice Theory}
                \item P. M. Cohn. {\em Universal Algebra}
                \item S. Vickers. {\em Topology via Logic}
	\end{enumerate}
\end{gelaber}

\subsection*{History and Motivation}

\begin{itemize}
	\item 1883: Cantor needed a well-order of $\mathbb R$, and considered the existence of such order as a ``Denkgesetz''.
	\item 1904: Zermelo proves that every set can be well-ordered, ($\WO$). Zermelo used $\AC$.

		$\ZF\vdash\AC\leftrightarrow\WO$
	\item Peano (1890) in a paper about Diff. Eq. explicitly avoids to use $\CC$ by using an algorithmic proof instead.
	\item $\ge$1904, Zermelo's paper prompted the so-called ``Grundlagenkrise''.
	\item 1905: Hamel proved with $\WO$ the existence of a basis for $\mathbb R$ as a $\mathbb Q$-vector space and he used this result to give the general solution of the functional equation $f(x+y)=f(x)+f(y)$ ($f\colon \mathbb R\to \mathbb R$)
	\item $\WO$ made possible the use of transfinite induction ($\TI$).
	\item Zorn (1935) put forward Zorn's Lemma ($\ZL$), to make proofs shorter and more algebraic. (Kuratowski already introduced $\ZL$ in 1922)
	\item Teichmüller (1939) and Tukey (1940), Teichmüller-Tukey Principle ($\TT$)
	\item Of course we know $\AC$, $\TT$, $\ZL$, $\WO$ are equivalent.
	\item Raoult (1988): {\em Open Induction} ($\OI$), equivalent to $\ZL$ and makes proofs even shorter.
	\item Coquand, Bergen (2004): Dependent choice can be replaced by a combinatorial form of $\OI$.
	\item $\AC$ is problematic from a constructive point of view. 

	$\AC + \Pow\vdash \EM$ (Dizconescu, ~1970) ($\EM$ = Law of excluded middle ($\forall_x (P(x)\lor\neg P(x))$), $\Pow$ = Powerset axiom)
	\item Gödel 1940: $\ZF \not\vdash \bot \to \ZF\not\vdash \neg \AC$
	\item Cohen 1963: $\ZF \not\vdash \bot \to \ZF\not\vdash \AC$
	\item $\OI$ is an alternative to $\AC$.
	\item Hilbert's Programme (HP): Justify the use of ideal objects (e.g. objects constructed by means of $\ZL$ or $\AC$) and transfinite methods.
		Prove with finite methods, that the use of idealistic methods is consistent.
	\item Revised form of HP (Kreisel and Feferman): Eliminate the use of ideal objects and use only finite and constructive proof methods.
	\item Successful for a considerable part of commutative algebra (Lombardi, Coquand)
\end{itemize}

{\bf Preliminaries (1).}
\begin{itemize}
	\item {\em Partial order} $\le$ (reflexive, transitive, antisymmetric), $(X,\le)$ is a {\em poset}.
	\item A {\em chain}, or {\em total order} or {\em linear order} is a partial order satisfying $x\le y\lor y\le x$
	\item On a poset $(X,\le)$ we talk about minimal/maximal elements. e.g. $x$ is {\em minimal} in $X$ $\iff$ $\forall_{y\in X}(y\le x\to y=x)$
		or equiv. $\neg \exists_{y\in X}(y\le x \land y\neq x)$
	\item $(X,\le)$ is a chain, $x$ is minimal (maximal), then we say: $x$ is the {\em least} ({\em greatest}) element.
	\item $\le$ {\em well-founded}: every non-empty subset has a minimal element.
	\item $\le$ {\em well-order}: $\le$ is well-founded linear order.
	\item $\WO$: every set can be well-ordered.
	\begin{example}
		\begin{enumerate}[(i)]
			\item $\mathbb N$ is well-ordered by $\le$
			\item $\mathbb Q^0_+ = [0,+\infty)\cap \mathbb Q$. It is linearly ordered, has least element (0), but it is not well-founded. 
				($s=(\sqrt{2}, +\infty)\cap \mathbb Q$).
			\item Transfinite Induction ($\TI$) on a poset $X$. Every progressive subset $S$ of $X$ equals $X$.
				\[\underbrace{\forall_x[\forall_{y<x}(y\in S)\to(x\in S)]}_{S-\text{progressive}}\to \underbrace{\forall_x(x\in S)}_{X=S} \]
			\item If $\le$ is well-founded order, then $\TI$ holds on $(X,\le)$.
			[If $S$ progressive and $S\neq X$, then $R=X-S$ is non-empty and therefore it has a minimal element $x$, so that $x\in S$ since $S$ is progressive. $\lightning$.]
			\item On $\mathbb N$, $\TI$ rewrites as: $\forall_n[\forall_{m<n}(m\in S)\to n\in S]\to \forall_n(n\in S)$.
		\end{enumerate}
	\end{example}
	\end{itemize}

\begin{theorem}[Hamel, 1905]
	Any linearly independent subset $S\in V$, when $V$ is a vector space over $\mathbb K$ can be extended to a base $S'\supset S$.
	\begin{proof}
		Consider a well-order on $V$, $\langle V_\alpha\mid \alpha\le \bar\alpha\rangle$ ($\bar\alpha$ ordinal corresp. to the well-order on $V$)
		We can define a (partial) function $f\colon\bar\alpha\to V$:
		\[\text{$f(\alpha)$ = the least element of $V$ that is not a linear combination of $f(\beta)$ with $\beta<\alpha$ in $S$}\]
		Of course $f$ is not defined, if such an element does not exist.
		\begin{itemize}
			\item $f$ injective
			\item $f(\bar\alpha)\cup S$ is linearly independent. 
				Suppose that a finite linear combination of elements of $S$ and values of $f$ equals $0$, and we can assume all coefficients to be non-zero.
				This combination must induce some element of $f(\bar\alpha)$, and let $\alpha_0$ the maximal of the ordinals encountered.
				Then $f(\alpha_0)$ is a linear combination of $S$ and elements of the form $f(\beta)$ $\beta<\alpha_0$. $\lightning$.
		\end{itemize}
		Since $\bar\alpha$ has the cardinality of $V$, $f$ is defined as an initial segment of kind $[0,\alpha)$, with $f(\alpha)$ undefined.
		This means precisely that every element of $V$ is linear combination of $S$ and $(f(\beta)\colon\beta<\alpha)$ 
	\end{proof}
\end{theorem}

\begin{gelaber}
	In 1821: Cauchy addressed the following functional equation:
	\[f(x+y) = f(x)+f(y) \qquad f\colon \mathbb R\to \mathbb R\]
	Cauchy proved that all the continuous solutions are linear, of the form $f(x)=c\cdot x$ for some $c\in \mathbb R$.
	Hamel first proved that $\mathbb R$ has a $\mathbb Q$-basis.

	Suppose $f\colon \mathbb R\to \mathbb R$ is additive. Then:
	\begin{itemize}
		\item $f(x_1+\ldots+x_n)=f(x_1)+\ldots+f(x_n)$
		\item $f(n\cdot x) = n\cdot f(x)$ for all $n\in \mathbb N$
		\item Since $f(0)= f(0+0)=f(0)+f(0) \to  f(0)=0$.
			Hence if $n\le 0$, $0=f(nx+(-n)x)=f(nx)-nf(x)$.
			So $f(nx) = nf(x)$ for all $n\in \mathbb Z$.
		\item If $q=\frac{m}{n}\in \mathbb Q$, then $n\cdot q=m$ so that $n\cdot f(q)=m\cdot f(i)$, so that, posing $c= f(i)$, we have $f(q)=c\cdot q$.
	\end{itemize}
	If $f$ is continuous, then $f(x)=c\cdot x$ for all $x\in \mathbb R$. (Cauchy's Result).
	If $x$ is real, $y=\frac{m}{n}\cdot x$, then $f(n\cdot y)= f(m\cdot x) \leadsto f(y) = \frac{m}{n} f(x)$.
	Hence $f$ is $\mathbb Q$-linear.
	If we have a basis of $\mathbb R$ over $\mathbb Q$, say $B$, then each $h$ is determined by its values on $B$.
\end{gelaber}

\begin{theorem}
	If $f\colon \mathbb R\to\mathbb R$ is a non-continuous solution $f$ of the Cauchy equation, then it's graph $G(f)=\{(x, f(x))\colon x\in \mathbb R\}$ is dense in $\mathbb R^2$
	\begin{proof}
		Let $(x,y)\in \mathbb R^2$ and $U$ is a neighborhood of $(x,y)$.
		Since $f$ is a non-$\mathbb R$-linear solution, there exist $a,b\neq 0$ in $\mathbb R$, such that $\alpha=\frac{f(a)}{a}$ and $\beta= \frac{f(b)}{b}$ are different.
		This means $u=(a,f(a)), v=(b,f(b))$ are independent, and therefore are a basis of $\mathbb R$.
		There exist $p,q\in \mathbb R$ sucht that $(x,y)=pu+qv$.
		Since $\bar{\mathbb Q^2}= \mathbb R^2$, we can find $\bar p, \bar q \in \mathbb Q$ such that $\bar pu+\bar qu \in U$. 
		Therefore $\bar pu+\bar pv= (\bar pa + \bar qb, \bar pf(a)+\bar qf(b)) = (\bar pa+\bar qb, f(\bar pa+ \bar qb))\in U\cap G(f)$
	\end{proof}
\end{theorem}

{\bf Preliminaries (2):} Zorn's Lemma.
Let $(X,\le)$ be a poset, $S\subseteq X$, $x\in X$. 
\begin{itemize}
	\item $x$ an {\em upper bound} of $S$: $\forall_{s\in S}(s\le x)$.
	\item $x$ {\em least upper bound} or {\em supremum} of $S$: $\forall_{u\in X}[\forall_{s\in S}(s\le u) \leftrightarrow x\le u]$, that is:
	\begin{enumerate}[(i)]
		\item $x$ is an upper bound of $S$. ($x=u$, $\leftarrow$).
		\item if $u\in X$ upper bound of $S$, then $x\le u$ ($\rightarrow$).
	\end{enumerate}
	\item {\em Common form of Zorn's Lemma}: If $X\neq \emptyset$ and every chain $C\subseteq X$ with $C\neq \emptyset$ has an upper bound, then $X$ has a maximal element.
	\item We could chop $X\neq\emptyset$ together with $X\neq \emptyset$, $[\emptyset \text{is chain}]$, or we can keep $X\neq \emptyset$ and every chain $C\subseteq X$, $C\neq\emptyset$ has a supremum.
	\item All of this can be reversed: 
		Let $(X,\le)$ be a poset.
		$D\subseteq X$ is called {\em directed}: $\forall_{x,y \in D} \exists_{z\in D}(x\le z \land y\le z)$.
	\item Every chain is a directed subset.
	\item A maximal element of a directed subset is also its greatest element.
	\item $X$ {\em directed complete}: every directed subset $D\subseteq X$, with $D\neq \emptyset$, $D$ has a supremum in $X$, we write the supremum as $\bigvee D$.
	\item {\em dcpo}: directed complete partial order.
	\item $V$ Vectorspace, $S$ Subspace. $V, S =\{ W\colon W\le V\}$ is a dcpo with $\le$ as partial order with $V$ as $V$. Exercise!
	
	\item A subset of a dcpo $X$ is {\em closed} if $\bigvee D\in S$ for all $D\subseteq S$ non-empty directed subset.
	\item $S$ is closed subset of the dcpo $(\mathbb P, \subseteq)$
	\item Here follow two equivalent formulations of Zorn's Lemma:
	\begin{itemize}
		\item Every dcpo $X\neq \emptyset$ has a maximal element
		\item If $X$ is a dcpo, then every closed subset $S\subseteq X$ with $S\neq \emptyset$ has a maximal element.
	\end{itemize}
\end{itemize}

\marginpar{\tiny{13.06.2013}}

\begin{definition}
	Sei $(x,\le)$ partielle Ordnung, $D\subseteq X$ {\em gerichtet}, wenn jede endliche Teilmenge von $D$ eine obere Schranke in $D$ hat.
	Dies ist gleichbedeutend mit $D\neq\emptyset$ und erfüllt die alte Definition, d.h. $\forall_{x,y\in D}\exists_{z\in D}(x\le z \land y\le x)$.
\end{definition}

\begin{lemma}[(Kuratowski-)Zorn (ZL)]
	Jeder dcpo $X\neq \emptyset$ hat ein maximales Element.
	Äquivalent: Ist $X$ ein dcpo und $S\subseteq X$ abgeschlossen, $S\neq \emptyset$, so hat $S$ ein max. Element
\end{lemma}

\begin{definition}
	Nun sei $S$ eine Menge; $X=\mathbb P(S)$ mit $\subseteq$; $F,G\subseteq X$.
	$F$ heißt {\em von endlichem Charakter}, wenn für alle $T\subseteq S$ gilt: $T\in F\iff \forall T_0\subseteq T(T_0 \text{ endlich } \to T_0\in F)$.
	$G$ {\em von coendlichem Charakter}, wenn für alle $T\subseteq S$ gilt: $T\in G\iff \exists_{T_0\subseteq T}(T_0 \text{ endlich } \land T_0\notin G)$.
	Falls $X=F\dot\cup G$, so gilt: $F$ von endlichem Charakter $\iff$ $G$ von coendlichem Charakter.
\end{definition}

\begin{lemma}[(Teichmüller-)Tukey (TuL)]
	Ist $S$ eine Menge, und $F\subseteq\mathbb P(S)$, so gilt: $F\neq\emptyset \land F$ von endlichem Charakter $\to$ $F$ hat maximales Element.
\end{lemma}

\begin{definition}
	Wieder sei $X$ dcpo. 
	$F\subseteq X$ {\em abgeschlossen}, wenn für jedes gerichtete $D\subseteq X$ gilt: 
	$\underbrace{D\subseteq F}_{\forall_{x\in X}(x\in D\to x\in F)}\to \bigvee D\in F$.
	$G$ {\em offen}, wenn für jedes gerichtete $D\subseteq X$ gilt: 
	$\bigvee D\in G\to \underbrace{\exists_{x\in X}(x\in D \land x\in G)}_{D\cap G\neq 0}$
	($X=F\dot\cup G\to F \text{ abg. } \iff G \text{ offen}$)
\end{definition}

\begin{lemma}
	Es sei $X=\mathbb P(S)$, $F,G\subseteq X$.
	\begin{enumerate}[(a)]
		\item $F$ von endlichem Charakter $\to$ $F$ abgeschlossen.
		\item $G$ von coendlichem Charakter $\to$ $G$ offen.
	\end{enumerate}
	\begin{proof}
		nur (a). Es sei $D\subseteq X$ gerichtet mit $D\subseteq F$.
		Zu Zeigen: $\bigcup D\in F$. 
		Es sei $T=\bigcup D$ und $T_0\subseteq T$, $T_0$ endl.
		Dazu gibt es endl. $D_0\subseteq D$ mit $T_0\subseteq\bigcup D_0$.
		Da $D$ gerichtet ist, hat $D_0$ eine obere Schranke $R\in D$.
		Dann $T_0\subseteq R\in F$, also $T_0\in F$, da $T_0$ endl. und $F$ von endl. Charakter.
	\end{proof}
\end{lemma}

\begin{definition}
	Sei $X$ wieder ein dcpo, $G\subseteq X$.
	$G$ progressiv, wenn $\forall_{x\in X} [\forall_{y>x}(y\in G)\to x\in G$
\end{definition}

\begin{definition}[Offene Induktion (OI)]
	Ist $X$ ein dcpo und $G\subseteq X$ offen, so gilt: $G$ progressiv $\to$ $G=X$, d.h.
	\[\forall_x[\forall_{y>x}(y\in G)\to x\in G]\to \forall_{x\in X}(x\in G)\]
	$OI$ ist $TI$ für offene $G\subseteq X$ mit $X$ dcpo.
\end{definition}

\begin{definition}[Tukey-Induktion (TuI)]
	Ist $S$ Menge, $G\subseteq\mathbb P(S)$, so gilt:
	$G$ von coendl. Charakter $\land$ $G$ progressiv $\to$ $G=\mathbb P(S)$
\end{definition}

\begin{theorem}
	\begin{enumerate}[(a)]
		\item $\ZL \iff \OI$
		\item $\TuL \iff \TuI$
	\end{enumerate}
	\begin{proof}
		Nur (a).
		$X$ dcpo, $X=F\dot\cup G$, dann: $F=\emptyset\iff G=X$, 
		$F$ abgeschlossen $\iff$ $G$ offen;
		$F$ hat kein max. El. $\iff$ $G$ progressiv.
		
		$\ZL$ für $X$ auch als: $S\subseteq X$ abgeschlossen, hat kein maximales Element $\to$ $S=\emptyset$.
		$\OI$ für $X$: $G\subseteq X$ offen, progressiv $\to$ $G=X$.
	\end{proof}
\end{theorem}

\section*{Allgemeine Abhängigkeit}

\begin{definition}
	Es sei $S$ eine Menge, sowie $\lhd\subseteq S\times \mathbb P(S)$.
	Stets seien $a, b, c\in S$ und $U, V, W\in \mathbb P(S)$.
	$\lhd$ {\em Überdeckung(srelation)}, wenn gelten:
	\begin{itemize}
		\item {\em Reflexivität}: $a\in U\to a\lhd U$
		\item {\em Transitivität}: $ a\lhd U \land U\lhd V \to a\lhd V$
	\end{itemize}
	Wobei $U\lhd V$ steht für $\forall_{b\in U}(b\lhd V)$. %%%
	
\end{definition}

\begin{remark}
	Eine Überdeckungsrelation ist das gleiche wie ein {\em Abschlußoperator} $U\mapsto U^{\lhd}$ auf $\mathbb P(S)$, mit den folgenden Axiomen:
	\begin{itemize}
		\item {\em Reflexivität}: $U\subseteq U^\lhd$
		\item {\em Transitivität}: $U\subseteq V^\lhd \to U^\lhd \subseteq V^\lhd$ % endspricht CUT in kalkülen -> schlecht
	\end{itemize}
	Korrespondenz $\lhd \leftrightsquigarrow \_^{\lhd}$:
	Zu $\lhd$ definiere $U^\lhd= \{a\in S: a\lhd U\}$.
	$a\lhd U \leftrightsquigarrow a\in U^\lhd$.

	Alternatives Axiomensystem:
	\begin{itemize}
		\item Reflexivität: wie oben.
		\item Monotonie: $U\subseteq V \to U^\lhd \subseteq V^\lhd$
		\item Idempotenz: $U^{\lhd\lhd} \subseteq U^{\lhd}$. (mit Refl. sogar $=$)
	\end{itemize}
	[R+T $\to$ M; T$\to$I; M+I$\to$T]
\end{remark}

\begin{definition}
	Eine Überdeckungsrelation $\lhd$ heißt
	\begin{itemize}
		\item {\em unitär} oder {\em Schottsch}, wenn aus $a\lhd U$ folgt: $\exists_{b\in U}(a\lhd\{b\})$.
		\item {\em finitär} oder {\em Stonesch}, wenn aus $a\lhd U$ folgt: $\exists_{U_0\subseteq U} (U_0 \text{ endlich} \land a\lhd U_0)$.
	\end{itemize}
	Eine finitäre Überdeckungsrelation $\lhd$ heißt {\em Abhängigkeitsrelation}, wenn $\lhd$ die {\em Abhängigkeitseigenschaft} hat, d.h.
	wenn für alle $a,b\in S$, $U\subseteq S$ gilt: 
	\[ a\lhd U\cup\{b\}\to a\lhd U\lor b\lhd U\cup\{a\}\]
	Ein $U\subseteq S$ heißt {\em ($\lhd$-)abhängig}, wenn $\exists_{b\in U}(b\lhd U-\{b\})$.
	
	$U$ heißt {\em ($\lhd$-)unabhängig}, wenn  $\forall_{b\in U}(b\nlhd U-\{b\})$.
\end{definition}

\begin{remark}
	$U$ abhängig $\to$ $U\neq\emptyset$. Außerdem ist $\emptyset$ unabhängig.
\end{remark}

\begin{example}
	\begin{enumerate}[(a)]
		\item $S$ Menge; $a\lhd U \equiv a\in U$, d.h. $U^\lhd=U$; Dann $\lhd$ unitär und jedes $U\subseteq S$ ist unabhängig.
		\item $S$ Vektorraum; $U^\lhd=(U)$ der von $U$ erzeugte Untervektorraum. $\lhd$ unitär; ``(un)abhängig'' ist ``linear (un)abhängig'' (!)
		\item $R\subseteq S$ komm. Ringe; für $U\subseteq S$ sei $R[U]$ die {\em Ringadjunktion} von $U$ an $R$ in $S$, d.h.
			\[ R[U]= \bigcup_{n\ge 0} \{ f(u_1,\ldots,u_n)\colon f\in R[X_1,\ldots,X_n]; u_1,\ldots, u_n \in U\} \]
			$U^\lhd = \bar{ R[U] }^S$ ganzer Abschluß von $R[U]$ in $S$, d.h. $a\lhd U\iff a^n = r_1a^{n-1}+\ldots+r_{n-1} a + r_n$ für geeignete $n\ge 1$; $r_1,\ldots,r_n\in R[U]$.
			
			Dann: $\lhd$ finitäre Überdeckungsrelation.
			$R, S$ Körper, $R(U)$ statt $R[U]\to \lhd$ Abhängigkeitsrelation und ``$\lhd$-(un)abhängigkeit'' ist ``algebraisch (un)abhängig''.
			Siehe B.L. van der Warden, (Moderne) Algebra I, §64.
	\end{enumerate}
\end{example}

\begin{definition}
	Nun sei $\lhd$ wieder eine allgemeine Abhängigkeitsrelation.
	$U$ {\em erzeugt} $S$, wenn $S\lhd U$, d.h. $\forall_{b\in S}(b\lhd U)$.
	$U$ {\em Basis}, wenn $U$ unabhängig, und $U$ erzeugt $S$.
	
	In den Beispielen (b) und (c) ist eine Basis eine Vektorraumbasis bzw. eine Tanszendenzbasis.
	$U$ ist {\em maximal unabhängig} genau dann, wenn jedes $V\subseteq S$ mit $V\supsetneq U$ abhängig ist (*), sowie $U$ unabhängig ist.
\end{definition}

\begin{lemma}
	$U$ maximal unabhängig $\iff$ $U$ Basis
	\begin{proof}
		``$\Rightarrow$`` (gilt i.a. nicht für $\lhd$ nur Überdeckungsrelation):
		Zeige: (*) $\land$ $S\nlhd U \to U \text{ abhängig}$.
		Nehme $b\in S$ mit $b\nlhd U$.
		Speziell $b\notin U$, d.h. $U\subsetneq U\cup \{b\}$.
		Nach (*) ist $U\cup \{b\}$ abhängig, d.h. es gibt $a\in U\cup \{b\}$ mit $a\lhd (U\cup \{b\})-\{a\}$.
		Sofort folgt $a\neq b$ [$a=b\to b\lhd U$ $\lightning$].
		Also $a\lhd (U-\{a\})\cup\{b\}$.
		Nach Abhängigkeitseigenschaft ist dann $a\lhd U-\{a\}$, damit $U$ abhängig, da ja $a\in U$ wegen $a\neq b$, 
		oder $b\lhd (U-\{a\})\cup\{a\}$, d.h. $b\lhd U$. $\lightning$.

		``$\Leftarrow$'': Nur zu Zeigen: (*).
		Ist $V\supsetneq U$, etwa $b\in V-U$, so ist $U-\{b\}=U$ und $b\lhd U$, also $b\lhd U-\{b\}$ und damit $b\lhd V-\{b\}$, also $V$ abhängig.
	\end{proof}
\end{lemma}

% Mail an pschust@maths.leeds.ac.uk Betreff: transfinit

\begin{theorem}
	Zu jedem unabhängigen $U\subseteq S$ gibt es eine Basis $W$ mit $W\supseteq U$.
	\begin{proof}[Beweis mit ZL]
		Verwende obiges Lemma.
		Es sei $G=E\cap F$, wobei $E=\{V\subseteq S\colon V\supseteq U\}$ und 
		$F = \{V\subseteq S\colon V \text{unabhängig}\}$ ein maximales Element $W$ von $G$ ist die gewünschte Basis.
		Zu Zeigen: $G\neq \emptyset$ und $G$ dcpo.
		Nun ist $U\in G$.
		Ist $D\subseteq G$ gerichtet, so ist $\bigcup D\in E$, da $D\neq \emptyset$, sowie $\bigcup D\in F$, da $F$ abgeschlossen, da $F$ von endl. Charakter (Lemma oben).
	\end{proof}
\end{theorem}

\begin{gelaber}
	{\bf Typische Anwendung} $V$ Vektorraum, $x\in V$. 
	Dann gilt: $\forall_{\varphi \in V^*} ( \varphi(x)=0)\to x=0$.
	\begin{proof}[Indirekter Beweis, mit ZL]
		Wäre $x\neq 0$, so wäre $U=\{x\}$ unabhängig, also gäbe es (Satz) eine Basis $W$ von $V$ mit $W\supseteq U$; d.h. $x\in W$.
		Definiere $\varphi \in V*$ durch $\varphi(x)=1$ und $\varphi\upharpoonright W-\{x\}=0$.
		Dann $\varphi(x)\neq 0$.$\lightning$.
	\end{proof}
\end{gelaber}

%20. Juni, 1. Hälfte

\marginpar{\tiny{20.06.2013}}

$V$ Vektorraum über $K$, $\bigcap\limits_{\varphi\in V^*} \ker(\varphi)=\{0\}$ (*)

{\bf Direkter Beweis von (*) mit OI:} $U, W, \ldots$: Untervektorräume [OI: $X$ dcpo, $G\subseteq X$ offen und progressiv $\Rightarrow G=X$]. $E, F,
\ldots$: endlich Erzeugte Untervektorräume. $X$ bestehe aus allen Untervektorräumen. $X$ partielle Ordnung mit $\subseteq$, sogar dcpo mit $\bigvee
D=\bigcup D$ für $D\subseteq X$ gerichtet.

{\bf Beachte:} $\bigcup D\in X$, da $D\subseteq X$ gerichtet. [Nun: zu $x,y\in \bigcup D$, das heißt $x\in U\in D$, $y\in W\in D$, also gibt es $Z\in
  D$ mit $U\subseteq Z, W\subseteq Z$. Dafür: $x, y\in Z$, aber $x+y\in Z$, weshalb $x+y\in\bigcup D$. Ferner ist $\lambda\cdot x\in U$, also
  $\lambda\cdot x\in\bigcup D$. Warum $0\in\bigcup D$? Nun: $D\neq\emptyset$, etwa $T\in D$, wofür $0\in T$, also $0\in\bigcup D$.

(*) kann man schreiben wie folgt: Für $x\in V$ gilt: $x\in\bigcap\limits_{\varphi\in U^*}\ker(\varphi) \Rightarrow \underbrace{\forall U\in X: x\in
    U}_{\mbox{d.h. }x=0}$.

Es sei $G=\{U\in X\mid x\in U\}$. Zu zeigen: für festes $x\in V$ gilt $x\in\bigcap\limits_{\varphi\in V^*}\ker(\varphi)\Rightarrow G=X$. Nach OI zu
zeigen: (1) $G$ offen, (2) $G$ progressiv.

(1) zu zeigen: $\bigcup D\in G\Rightarrow D\cap G\neq\emptyset$. Nun: $\bigcup D\in G$ heißt $x\in\bigcup D$, das heißt es gibt $W\in D$ mit $x\in W,
W\in G$.

(2) sei $U\in X$ mit $W\in G$ für jedes $W\in X$ mit $W\supsetneq U$. Zeige $U\in G$.

Fall 1: $U=V$. Dann $x\in U$, das heißt $U\in G$.

Fall 2: $U\subsetneq V$. Nehme $y\in V\backslash U$. Dann $U(y)=U+Ky$.

Fall 2.1: $U(y)=V$. Dann sogar $U\oplus K(y)=V$ [$U\ni\lambda\cdot y\Rightarrow\lambda=0$, da $\lambda\neq 0\Rightarrow y\in U$ $\lightning$]. Definiere $\varphi\in V^*$ durch $\varphi\upharpoonright U=0$,
$\varphi(y)=1$. Nach Voraussetzung ist $x\in\ker\varphi=U$, also $U\in G$.

Fall 2.2: $U(y)\subsetneq V$. Nehme $z\in V\backslash U(y)$. Nun
$U\subsetneq U(y)$ und $U\subsetneq U(z)$, also (nach Ind.) $U(y)\in G$ und $U(z)\in G$, das heißt $x\in U(y)$ und $x\in U(z)$, also $x\in U$ (nach
Lemma unten), das heißt $U\in G$.

\begin{lemma}
  $z\not\in U(y)\Rightarrow U(y)\cap U(z)\subseteq U$
  \begin{proof}
    Ist $x\in U(y)\cap U(z)$, das heißt $x=u+\lambda y\land x=v+\mu z$ für $u,v\in U$, so ist $\mu = 0$ [aus $\mu\neq 0$ folgte
      $z=\mu^{-1}(x-v)=\mu^{-1}(\underbrace{u-v}_{u}+\lambda v)\in U(y)$ $\lightning$], also $x=v\in U$.
  \end{proof}
\end{lemma}

Beweis mit ZL des Satzes, dass für jede Abhängigkeitsrelation $\lhd$ auf einer Menge $S$ für $U, V\subseteq S$ gilt: $U, V$ unabhängig und $U, V$
äquivalent (das heißt $U\lhd V\land V\lhd U$) impliziert $U, V$ gleichmächtig (das heißt es gibt Bijektion $U\rightarrow V$) (§).

Verwende: Satz von Cantor-Bernstein-Schröder: Sind $U, V$ Mengen und gibt es injektive Abbildungen $U\rightarrow V$ und $V\rightarrow U$, so sind $U$
und $V$ gleichmächtig. Da (§) symmetrisch ist in $U$ und $V$, reicht es zu zeigen: es gibt eine bijektive Abbildung $U\rightarrow V^*$ mit
$V^*\subseteq V$.

Dazu sei $W=U\cap V$, eventuell $W=\emptyset$. Definiere $X=\{\varphi':U'\rightarrow V'\mid \varphi\mbox{ bijektiv}, W\subseteq U'\subseteq U,
W\subseteq V'\subseteq V, V'\cup(U\backslash U')\mbox{ unabhängig}\}$. Wegen $\operatorname{id}_W\in X$ ist $X\neq\emptyset$.

Ordne $X$ durch $\varphi'\le\varphi''\equiv \varphi'\subseteq\varphi''$, das heißt, dass $\varphi''$ fortsetzung von $\varphi'$ ist, das heißt mit
$\varphi':U'\rightarrow V'$ und $\varphi'':U''\rightarrow V''$ ist $U'\subseteq U''$, $V'\subseteq V''$ und $\varphi''\upharpoonright U'\rightarrow
V'$ ist gleich $\varphi'$. Nun ist $X$ ein dcpo. Ist $D\subseteq X$ gerichtet, so ist $\bar{\varphi}=\bigcup D$ Element von $X$ (!); haben: $\bar{\varphi}=\bigvee D$.

Zu (!): $\bar{\varphi}:\bar{U}\rightarrow\bar{V}$ mit $\bar{U}=\bigcup\{\dom(\varphi')\mid \varphi'\in D\}$; analog
$\bar{V}=\bigcup\{\codom(\varphi')\mid \varphi'\in D\}$; klar: $\bar{\varphi}$ bijektiv.

%20. Juni, 2. Hälfte
Zeige: $\bar{V}\cup(U\backslash\bar{U})$ ist unabhängig. Erinnerung: ``Unabhängig'' von endl. Char. Ist $F\subseteq \bar{V}\cap(U\backslash\bar{U})$,
$F$ endlich, so gibt es (da $D$ gerichtet) $\varphi':U'\rightarrow V'$ mit $\varphi'\in D$ und $F\subseteq \underbrace{V'\cup(U\backslash
  U')}_{\mbox{unabh.}}$, also $F$ unabhängig. Nach ZL hat $X$ ein maximales Element $\varphi^*:U^*\rightarrow V^*$. Noch zu zeigen: $U^*=U$. Annahme
$U^*\subsetneq U$. Nehme $u\in U\backslash U^*$. Nun: $R^*=V^*\cup(U\backslash U^*)$. $R^*$ unabängig (da $\varphi^*\in X$), also $V\not\lhd
R^*\backslash\{u\}$ [sonst $u\lhd R^*\backslash\{u\}$ (da $U\lhd V$; Transitivität)], etwa $v\in V$ mit $v\not\lhd R^*\backslash\{u\}$. Betrachte
$R^{**})=(V^*\cup\{v\})\cup(U\backslash(U^*\cap\{u\}))=(R^*\backslash\{u\})\cup\{v\}$. $R^{**}$ ist unabhängig, das heißt $\forall_{r\in
  R^{**}}(r\not\lhd R^{**}\backslash\{r\})$ [sonst, etwa $r\in R^\{**\}$ mit $r\lhd R^{**}\backslash\{r\}$, dann $r\neq v$ ($r=v: v\lhd
  R^{**}\backslash\{v\}=R^*\backslash\{u\}$ $\lightning$), also $r\in R^*\backslash\{u\}$, aber $r\lhd R^{**}\backslash\{r\}$ heißt damit $r\lhd
  (R^*\backslash\{u,r\})\cup\{v\}$, weshalb $r\lhd R^*\backslash\{r\}$ $\lightning$ ($R^*$ unabhängig) oder
  $v\lhd(R^*\backslash\{u,r\})\cup\{r\}=R^*\backslash\{u\}$ $\lightning$]. Also wäre $\varphi^{**}:U^*\cup\{u\}\rightarrow V^*\cup\{v\}$ mit
$\varphi^{**}(x) = \left\{ \begin{array}{rl} \varphi^*(x),& x\in U^* \\ v,& x = u \end{array}\right.$ in $X$ mit $\varphi^* < \varphi^{**}$ im
Widerspruch zur Maximalität von $\varphi^*$.

\section{Maximale Ideale}

\begin{gelaber}
  {\bf Erinnerung:} Ring $(R,+,\cdot,0,1)$ mit $(R,+,\cdot)$ abelsche Gruppe, $(R,\cdot,1)$ Halbgruppe, $x(y+z)=xy+xz$, $(x+y)z=xz+yz$. Stets sei $R$
  kommutativ, das heißt $xy=yx$ gilt in $R$. $x$ heißt {\bf Einheit} von $R$ oder {\bf invertierbar}, wenn es ein $y$ gibt mit $xy=1$. $R^*$ sei die
  Gruppe der Einheiten von $R$, $R^*$ abelsch. Ein Ring $R$ ist ein Körper, wenn $1\neq 0$ und $R^*=R\backslash\{0\}$. {\bf $R$-Modul}: $M$ ist eine
  abelsche Gruppe $(M,+,0)$ mit $R\times M\rightarrow M$, $(r,x)\mapsto rx$, sodass $r(sx)=(rs)x$, $1x=x$, $r(x+y)=rx+ry$, $(r+s)x=rx+sx$. {\bf
    $U\subseteq M$ ($R$-)Untermodul}: $U$ Untergruppe von $(M,+,0)$ derart dass gilt: $r\in R \land x\in U\Rightarrow rx\in U$. $Rx=\{rx\mid r\in R\}$,
  $U+V=\{u+v\mid u\in U, v\in V\}$. $(X_1,\ldots,X_n)=Rx_1+\ldots+Rx_n$. Jede abelsche Gruppe ist ein $\mathbb{Z}$-Modul via $n\cdot
  x=\underbrace{x\cdot x}_{n\mbox{-mal}}$, $(-n)x=-(nx)$, für $n\ge 0$. Ein {\bf Ideal} von $R$ ist ein $R$-Untermodul von $R$. Ein Ideal $M$ von $R$
  heißt {\bf maximales Ideal}, wenn $M\neq R$ und für jedes Ideal $I$ von $R$ gilt: $M\subseteq I \Rightarrow M=I \vee I=R$. Das heißt, $M$ ist
  maximal unter den Idealen $I$ mit $I\neq R$.
\end{gelaber}

\begin{theorem}
  Es sei $R$ ein Ring. Zu jedem Ideal $I$ von $R$ mit $I\neq R$ gibt es ein maximales Ideal $M$ von $R$ mit $M\supseteq I$.
\end{theorem}

\begin{corollary}
  Jeder Ring, in dem $1\neq 0$, hat ein maximales Ideal. [Satz mit $I=\{0\}$]
\end{corollary}

\begin{proof}[Beweis Satz mit ZL]
$X=\{F\subseteq R:F\mbox{ Ideal}, F\neq R, F\supseteq I\}$ ist $\neq\emptyset$ (da $I\in X$) und ein dcpo, denn ist $D\subseteq X$ gerichtet, so ist
  $\bigcup D$ ein Ideal (!) mit $\bigcup D\supseteq I$ (!) und $\bigcup D\neq R$ [sonst wäre $1\in \bigcup D$, etwa $1\in F\in D$, also $F=R$ nach
    Bemerkung unten $\lightning$]. Also: $\bigcup D\in X$. Nach ZL hat $X$ ein maximales Element, das heißt $R$ hat ein maximales Ideal $\supseteq I$.
\end{proof}

\begin{gelaber}
  {\bf Bemerkung:} Für jedes Ideal $F$ eines Rings $R$ sind äquivalent:
  \begin{itemize}
    \item[i.] $F=R$
    \item[ii.] $1\in F$
    \item[iii.] $R^*\cap F\neq\emptyset$
  \end{itemize}
  \begin{proof}
    $i.\Rightarrow ii.\Rightarrow iii.$ trivial. $iii.\Rightarrow i.$: Ist etwa $r\in R^*\cap F$, so gibt es $s\in R$ mit $rs=1$; für $t\in F$ gilt
    dann $t=tsr\in F$ weil $r\in F, sr=1$.
  \end{proof}
\end{gelaber}

\begin{gelaber}
  {\bf Variante von ZL}

  ZL': Ist $X$ ein dcpo, so gibt es zu jedem $x\in X$ ein maximales Element $y\in X$ mit $y\ge x$.

  ZL' $\Rightarrow$ ZL: trivial.

  ZL $\Rightarrow$ ZL': Es sei $x\in X$. Mit $X$ dcpo ist auch $\uparrow x = \{z\in X\mid z\ge x\}$ ein dcpo mit $\uparrow x \neq\emptyset$ (da
  $x\in\uparrow x$).
\end{gelaber}


\end{document}
