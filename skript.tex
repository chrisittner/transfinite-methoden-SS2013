%% -*- coding: utf-8; fill-column: 150 -*-
%\documentclass[headsepline=true]{scrartcl}
\documentclass[headsepline=true,DIV=11]{scrartcl}
\usepackage[utf8]{inputenc}
\usepackage[T1]{fontenc}
\usepackage[ngerman]{babel}
\usepackage[colorlinks=true,linkcolor=black,bookmarks]{hyperref}
\usepackage{amssymb, amsmath, amsthm, enumerate, verbatim, extarrows, marvosym, lmodern, stmaryrd, accents, bbm}

\pagestyle{headings}
\providecommand\thetheorem{}
\renewcommand{\thetheorem}{\arabic{theorem}}
%\newtheorem*{question}[theorem]{Frage}
\newtheorem*{theorem}{Satz}
\newtheorem*{lemma}{Lemma}
\newtheorem*{corollary}{Korollar}
\newtheorem*{remark}{Bemerkung}
\theoremstyle{definition}
\newtheorem*{definition}{Definition}
\newtheorem*{example}{Beispiel}
\newtheorem*{convention}{Konvention}
\newtheorem*{motivation}{Motivation}
\newtheorem*{notation}{Notation}
\newtheorem*{question*}{Frage}
\newenvironment{gelaber}{}{}
\newenvironment{preamble}{}{}
\setlength{\parindent}{0in}
\renewcommand{\bar}[1]{\overline{#1}}
%\renewcommand{\Section}[1]{\section{#1} \setcounter{theorem}{0}\setcounter{remark}{0}}

\newcommand{\Card}{\operatorname{Card}}
\newcommand{\Char}{\operatorname{Char}}
\newcommand{\WO}{\operatorname{WO}}
\newcommand{\AC}{\operatorname{AC}}
\newcommand{\ZF}{\operatorname{ZF}}
\newcommand{\ZFC}{\operatorname{ZFC}}
\newcommand{\TI}{\operatorname{TI}}
\newcommand{\TT}{\operatorname{TT}}
\newcommand{\ZL}{\operatorname{ZL}}
\newcommand{\CC}{\operatorname{CC}}
\newcommand{\OI}{\operatorname{OI}}
\newcommand{\Pow}{\operatorname{Pow}}
\newcommand{\EM}{\operatorname{EM}}
\newcommand{\TuL}{\operatorname{TuL}}
\newcommand{\TuI}{\operatorname{TuI}}
\newcommand{\dom}{\operatorname{dom}}
\newcommand{\codom}{\operatorname{codom}}
\newcommand{\nlhd}{\ntriangleleft} % Negation von \ldh
\newcommand{\Max}{\operatorname{Max}}
\newcommand{\Spec}{\operatorname{Spec}}
\newcommand{\Irr}{\operatorname{Irr}}
\newcommand{\Sat}{\operatorname{Sat}}
\newcommand{\Jac}{\operatorname{Jac}}
\newcommand{\Ker}{\operatorname{Ker}}
\newcommand{\id}{\operatorname{id}}

\begin{document}
\begin{preamble}
\subject{Vorlesung aus dem Sommersemester 2013}
\title{Transfinite Beweismethoden}
\author{Dr. Schuster}
\date{}
%\publishers{\small ge\TeX{}t von }

\maketitle
%\thispagestyle{empty}
%\tableofcontents
%\clearpage
\end{preamble}

\setcounter{section}{0}

\marginpar{\tiny{6.06.2013}}

\subsection*{References}

\begin{gelaber}
	\begin{enumerate}
		\item A. Kertesz. {\em Einführung in die Transfinite Algebra} (main reference)
		\item I. Kaplanski. {\em Set Theory and Metric Spaces}, AMS 2001
		\item G. H. Moore. {\em Zermelo's Axiom of Choice}, Springer
		\item T. Jech. {\em The Axiom of Choice}, North-Holland
		\item H. Rubin, J. E. Rubin. {\em Equivalents of the Axiom of Choice},  Vol. I + II
		\item M. Erne. {\em Einführung in die Ordnungstheorie}, 1982
		\item H. Herrlich. {\em Axiom of Choice}, Springer, 2006
		\item P. Howard, J.E. Rubin. {\em Consequences of the Axiom of Choice}, AMS, 1998
		\item J.L. Bell. {\em The Axiom of Choice}, College, 2009
                \item G. Birkhoff. {\em Lattice Theory}
                \item P. M. Cohn. {\em Universal Algebra}
                \item S. Vickers. {\em Topology via Logic}
	\end{enumerate}
\end{gelaber}

\subsection*{History and Motivation}

\begin{itemize}
	\item 1883: Cantor needed a well-order of $\mathbb R$, and considered the existence of such order as a ``Denkgesetz''.
	\item 1904: Zermelo proves that every set can be well-ordered, ($\WO$). Zermelo used $\AC$.

		$\ZF\vdash\AC\leftrightarrow\WO$
	\item Peano (1890) in a paper about Diff. Eq. explicitly avoids to use $\CC$ by using an algorithmic proof instead.
	\item $\ge$1904, Zermelo's paper prompted the so-called ``Grundlagenkrise''.
	\item 1905: Hamel proved with $\WO$ the existence of a basis for $\mathbb R$ as a $\mathbb Q$-vector space and he used this result to give the general solution of the functional equation $f(x+y)=f(x)+f(y)$ ($f\colon \mathbb R\to \mathbb R$)
	\item $\WO$ made possible the use of transfinite induction ($\TI$).
	\item Zorn (1935) put forward Zorn's Lemma ($\ZL$), to make proofs shorter and more algebraic. (Kuratowski already introduced $\ZL$ in 1922)
	\item Teichmüller (1939) and Tukey (1940), Teichmüller-Tukey Principle ($\TT$)
	\item Of course we know $\AC$, $\TT$, $\ZL$, $\WO$ are equivalent.
	\item Raoult (1988): {\em Open Induction} ($\OI$), equivalent to $\ZL$ and makes proofs even shorter.
	\item Coquand, Bergen (2004): Dependent choice can be replaced by a combinatorial form of $\OI$.
	\item $\AC$ is problematic from a constructive point of view. 

	$\AC + \Pow\vdash \EM$ (Dizconescu, ~1970) ($\EM$ = Law of excluded middle ($\forall_x (P(x)\lor\neg P(x))$), $\Pow$ = Powerset axiom)
	\item Gödel 1940: $\ZF \not\vdash \bot \to \ZF\not\vdash \neg \AC$
	\item Cohen 1963: $\ZF \not\vdash \bot \to \ZF\not\vdash \AC$
	\item $\OI$ is an alternative to $\AC$.
	\item Hilbert's Programme (HP): Justify the use of ideal objects (e.g. objects constructed by means of $\ZL$ or $\AC$) and transfinite methods.
		Prove with finite methods, that the use of idealistic methods is consistent.
	\item Revised form of HP (Kreisel and Feferman): Eliminate the use of ideal objects and use only finite and constructive proof methods.
	\item Successful for a considerable part of commutative algebra (Lombardi, Coquand)
\end{itemize}

{\bf Preliminaries (1).}
\begin{itemize}
	\item {\em Partial order} $\le$ (reflexive, transitive, antisymmetric), $(X,\le)$ is a {\em poset}.
	\item A {\em chain}, or {\em total order} or {\em linear order} is a partial order satisfying $x\le y\lor y\le x$
	\item On a poset $(X,\le)$ we talk about minimal/maximal elements. e.g. $x$ is {\em minimal} in $X$ $\iff$ $\forall_{y\in X}(y\le x\to y=x)$
		or equiv. $\neg \exists_{y\in X}(y\le x \land y\neq x)$
	\item $(X,\le)$ is a chain, $x$ is minimal (maximal), then we say: $x$ is the {\em least} ({\em greatest}) element.
	\item $\le$ {\em well-founded}: every non-empty subset has a minimal element.
	\item $\le$ {\em well-order}: $\le$ is well-founded linear order.
	\item $\WO$: every set can be well-ordered.
	\begin{example}
		\begin{enumerate}[(i)]
			\item $\mathbb N$ is well-ordered by $\le$
			\item $\mathbb Q^0_+ = [0,+\infty)\cap \mathbb Q$. It is linearly ordered, has least element (0), but it is not well-founded. 
				($s=(\sqrt{2}, +\infty)\cap \mathbb Q$).
			\item Transfinite Induction ($\TI$) on a poset $X$. Every progressive subset $S$ of $X$ equals $X$.
				\[\underbrace{\forall_x[\forall_{y<x}(y\in S)\to(x\in S)]}_{S-\text{progressive}}\to \underbrace{\forall_x(x\in S)}_{X=S} \]
			\item If $\le$ is well-founded order, then $\TI$ holds on $(X,\le)$.
			[If $S$ progressive and $S\neq X$, then $R=X-S$ is non-empty and therefore it has a minimal element $x$, so that $x\in S$ since $S$ is progressive. $\lightning$.]
			\item On $\mathbb N$, $\TI$ rewrites as: $\forall_n[\forall_{m<n}(m\in S)\to n\in S]\to \forall_n(n\in S)$.
		\end{enumerate}
	\end{example}
	\end{itemize}

\begin{theorem}[Hamel, 1905]
	Any linearly independent subset $S\in V$, when $V$ is a vector space over $\mathbb K$ can be extended to a base $S'\supset S$.
	\begin{proof}
		Consider a well-order on $V$, $\langle V_\alpha\mid \alpha\le \bar\alpha\rangle$ ($\bar\alpha$ ordinal corresp. to the well-order on $V$)
		We can define a (partial) function $f\colon\bar\alpha\to V$:
		\[\text{$f(\alpha)$ = the least element of $V$ that is not a linear combination of $f(\beta)$ with $\beta<\alpha$ in $S$}\]
		Of course $f$ is not defined, if such an element does not exist.
		\begin{itemize}
			\item $f$ injective
			\item $f(\bar\alpha)\cup S$ is linearly independent. 
				Suppose that a finite linear combination of elements of $S$ and values of $f$ equals $0$, and we can assume all coefficients to be non-zero.
				This combination must induce some element of $f(\bar\alpha)$, and let $\alpha_0$ the maximal of the ordinals encountered.
				Then $f(\alpha_0)$ is a linear combination of $S$ and elements of the form $f(\beta)$ $\beta<\alpha_0$. $\lightning$.
		\end{itemize}
		Since $\bar\alpha$ has the cardinality of $V$, $f$ is defined as an initial segment of kind $[0,\alpha)$, with $f(\alpha)$ undefined.
		This means precisely that every element of $V$ is linear combination of $S$ and $(f(\beta)\colon\beta<\alpha)$ 
	\end{proof}
\end{theorem}

\begin{gelaber}
	In 1821: Cauchy addressed the following functional equation:
	\[f(x+y) = f(x)+f(y) \qquad f\colon \mathbb R\to \mathbb R\]
	Cauchy proved that all the continuous solutions are linear, of the form $f(x)=c\cdot x$ for some $c\in \mathbb R$.
	Hamel first proved that $\mathbb R$ has a $\mathbb Q$-basis.

	Suppose $f\colon \mathbb R\to \mathbb R$ is additive. Then:
	\begin{itemize}
		\item $f(x_1+\ldots+x_n)=f(x_1)+\ldots+f(x_n)$
		\item $f(n\cdot x) = n\cdot f(x)$ for all $n\in \mathbb N$
		\item Since $f(0)= f(0+0)=f(0)+f(0) \to  f(0)=0$.
			Hence if $n\le 0$, $0=f(nx+(-n)x)=f(nx)-nf(x)$.
			So $f(nx) = nf(x)$ for all $n\in \mathbb Z$.
		\item If $q=\frac{m}{n}\in \mathbb Q$, then $n\cdot q=m$ so that $n\cdot f(q)=m\cdot f(i)$, so that, posing $c= f(i)$, we have $f(q)=c\cdot q$.
	\end{itemize}
	If $f$ is continuous, then $f(x)=c\cdot x$ for all $x\in \mathbb R$. (Cauchy's Result).
	If $x$ is real, $y=\frac{m}{n}\cdot x$, then $f(n\cdot y)= f(m\cdot x) \leadsto f(y) = \frac{m}{n} f(x)$.
	Hence $f$ is $\mathbb Q$-linear.
	If we have a basis of $\mathbb R$ over $\mathbb Q$, say $B$, then each $h$ is determined by its values on $B$.
\end{gelaber}

\begin{theorem}
	If $f\colon \mathbb R\to\mathbb R$ is a non-continuous solution $f$ of the Cauchy equation, then it's graph $G(f)=\{(x, f(x))\colon x\in \mathbb R\}$ is dense in $\mathbb R^2$
	\begin{proof}
		Let $(x,y)\in \mathbb R^2$ and $U$ is a neighborhood of $(x,y)$.
		Since $f$ is a non-$\mathbb R$-linear solution, there exist $a,b\neq 0$ in $\mathbb R$, such that $\alpha=\frac{f(a)}{a}$ and $\beta= \frac{f(b)}{b}$ are different.
		This means $u=(a,f(a)), v=(b,f(b))$ are independent, and therefore are a basis of $\mathbb R$.
		There exist $p,q\in \mathbb R$ sucht that $(x,y)=pu+qv$.
		Since $\bar{\mathbb Q^2}= \mathbb R^2$, we can find $\bar p, \bar q \in \mathbb Q$ such that $\bar pu+\bar qu \in U$. 
		Therefore $\bar pu+\bar pv= (\bar pa + \bar qb, \bar pf(a)+\bar qf(b)) = (\bar pa+\bar qb, f(\bar pa+ \bar qb))\in U\cap G(f)$
	\end{proof}
\end{theorem}

{\bf Preliminaries (2):} Zorn's Lemma.
Let $(X,\le)$ be a poset, $S\subseteq X$, $x\in X$. 
\begin{itemize}
	\item $x$ an {\em upper bound} of $S$: $\forall_{s\in S}(s\le x)$.
	\item $x$ {\em least upper bound} or {\em supremum} of $S$: $\forall_{u\in X}[\forall_{s\in S}(s\le u) \leftrightarrow x\le u]$, that is:
	\begin{enumerate}[(i)]
		\item $x$ is an upper bound of $S$. ($x=u$, $\leftarrow$).
		\item if $u\in X$ upper bound of $S$, then $x\le u$ ($\rightarrow$).
	\end{enumerate}
	\item {\em Common form of Zorn's Lemma}: If $X\neq \emptyset$ and every chain $C\subseteq X$ with $C\neq \emptyset$ has an upper bound, then $X$ has a maximal element.
	\item We could chop $X\neq\emptyset$ together with $X\neq \emptyset$, $[\emptyset \text{ is chain}]$, or we can keep $X\neq \emptyset$ and every chain $C\subseteq X$, $C\neq\emptyset$ has a supremum.
	\item All of this can be reversed: 
		Let $(X,\le)$ be a poset.
		$D\subseteq X$ is called {\em directed}: $\forall_{x,y \in D} \exists_{z\in D}(x\le z \land y\le z)$.
	\item Every chain is a directed subset.
	\item A maximal element of a directed subset is also its greatest element.
	\item $X$ {\em directed complete}: every directed subset $D\subseteq X$, with $D\neq \emptyset$, $D$ has a supremum in $X$, we write the supremum as $\bigvee D$.
	\item {\em dcpo}: directed complete partial order.
	\item $V$ Vectorspace, $S$ Subspace. $V, S =\{ W\colon W\le V\}$ is a dcpo with $\subseteq$ as partial order with $V$ as $V$. Exercise!
	
	\item A subset $S$ of a dcpo $X$ is {\em closed} if $\bigvee D\in S$ for all $D\subseteq S$ non-empty directed subset.
	\item $S$ is closed subset of the dcpo $(\mathbb P(V), \subseteq)$
	\item Here follow two equivalent formulations of Zorn's Lemma:
	\begin{itemize}
		\item Every dcpo $X\neq \emptyset$ has a maximal element
		\item If $X$ is a dcpo, then every closed subset $S\subseteq X$ with $S\neq \emptyset$ has a maximal element.
	\end{itemize}
\end{itemize}

\marginpar{\tiny{13.06.2013}}

\begin{definition}
	Sei $(x,\le)$ partielle Ordnung, $D\subseteq X$ {\em gerichtet}, wenn jede endliche Teilmenge von $D$ eine obere Schranke in $D$ hat.
	Dies ist gleichbedeutend mit $D\neq\emptyset$ und erfüllt die alte Definition, d.h. $\forall_{x,y\in D}\exists_{z\in D}(x\le z \land y\le x)$.
\end{definition}

\begin{lemma}[(Kuratowski-)Zorn (ZL)]
	Jeder dcpo $X\neq \emptyset$ hat ein maximales Element.
	Äquivalent: Ist $X$ ein dcpo und $S\subseteq X$ abgeschlossen, $S\neq \emptyset$, so hat $S$ ein max. Element
\end{lemma}

\begin{definition}
	Nun sei $S$ eine Menge; $X=\mathbb P(S)$ mit $\subseteq$; $F,G\subseteq X$.
	$F$ heißt {\em von endlichem Charakter}, wenn für alle $T\subseteq S$ gilt: $T\in F\iff \forall T_0\subseteq T(T_0 \text{ endlich } \to T_0\in F)$.
	$G$ {\em von coendlichem Charakter}, wenn für alle $T\subseteq S$ gilt: $T\in G\iff \exists_{T_0\subseteq T}(T_0 \text{ endlich } \land T_0\notin G)$.
	Falls $X=F\dot\cup G$, so gilt: $F$ von endlichem Charakter $\iff$ $G$ von coendlichem Charakter.
\end{definition}

\begin{lemma}[(Teichmüller-)Tukey (TuL)]
	Ist $S$ eine Menge, und $F\subseteq\mathbb P(S)$, so gilt: $F\neq\emptyset \land F$ von endlichem Charakter $\to$ $F$ hat maximales Element.
\end{lemma}

\begin{definition}
	Wieder sei $X$ dcpo. 
	$F\subseteq X$ {\em abgeschlossen}, wenn für jedes gerichtete $D\subseteq X$ gilt: 
	$\underbrace{D\subseteq F}_{\forall_{x\in X}(x\in D\to x\in F)}\to \bigvee D\in F$.
	$G$ {\em offen}, wenn für jedes gerichtete $D\subseteq X$ gilt: 
	$\bigvee D\in G\to \underbrace{\exists_{x\in X}(x\in D \land x\in G)}_{D\cap G\neq 0}$
	($X=F\dot\cup G\to F \text{ abg. } \iff G \text{ offen}$)
\end{definition}

\begin{lemma}
	Es sei $X=\mathbb P(S)$, $F,G\subseteq X$.
	\begin{enumerate}[(a)]
		\item $F$ von endlichem Charakter $\to$ $F$ abgeschlossen.
		\item $G$ von coendlichem Charakter $\to$ $G$ offen.
	\end{enumerate}
	\begin{proof}
		nur (a). Es sei $D\subseteq X$ gerichtet mit $D\subseteq F$.
		Zu Zeigen: $\bigvee D=\bigcup D\in F$. 
		Es sei $T=\bigcup D$ und $T_0\subseteq T$, $T_0$ endl.
		Dazu gibt es endl. $D_0\subseteq D$ mit $T_0\subseteq\bigcup D_0$.
		Da $D$ gerichtet ist, hat $D_0$ eine obere Schranke $R\in D$.
		Dann $T_0\subseteq R\in F$, also $T_0\in F$, da $T_0$ endl. und $F$ von endl. Charakter.
	\end{proof}
\end{lemma}

\begin{definition}
	Sei $X$ wieder ein dcpo, $G\subseteq X$.
	$G$ progressiv, wenn $\forall_{x\in X} [\forall_{y>x}(y\in G)\to x\in G]$
\end{definition}

\begin{definition}[Offene Induktion (OI)]
	Ist $X$ ein dcpo und $G\subseteq X$ offen, so gilt: $G$ progressiv $\to$ $G=X$, d.h.
	\[\forall_x[\forall_{y>x}(y\in G)\to x\in G]\to \forall_{x\in X}(x\in G)\]
	$OI$ ist $TI$ für offene $G\subseteq X$ mit $X$ dcpo.
\end{definition}

\begin{definition}[Tukey-Induktion (TuI)]
	Ist $S$ Menge, $G\subseteq\mathbb P(S)$, so gilt:
	$G$ von coendl. Charakter $\land$ $G$ progressiv $\to$ $G=\mathbb P(S)$
\end{definition}

\begin{theorem}
	\begin{enumerate}[(a)]
		\item $\ZL \iff \OI$
		\item $\TuL \iff \TuI$
	\end{enumerate}
	\begin{proof}
		Nur (a).
		$X$ dcpo, $X=F\dot\cup G$, dann: $F=\emptyset\iff G=X$, 
		$F$ abgeschlossen $\iff$ $G$ offen;
		$F$ hat kein max. El. $\iff$ $G$ progressiv.
		
		$\ZL$ für $X$ auch als: $S\subseteq X$ abgeschlossen, hat kein maximales Element $\to$ $S=\emptyset$.
		$\OI$ für $X$: $G\subseteq X$ offen, progressiv $\to$ $G=X$.
	\end{proof}
\end{theorem}

\section*{Allgemeine Abhängigkeit}

\begin{definition}
	Es sei $S$ eine Menge, sowie $\lhd\subseteq S\times \mathbb P(S)$.
	Stets seien $a, b, c\in S$ und $U, V, W\in \mathbb P(S)$.
	$\lhd$ {\em Überdeckung(srelation)}, wenn gelten:
	\begin{itemize}
		\item {\em Reflexivität}: $a\in U\to a\lhd U$
		\item {\em Transitivität}: $ a\lhd U \land U\lhd V \to a\lhd V$
	\end{itemize}
	Wobei $U\lhd V$ steht für $\forall_{b\in U}(b\lhd V)$. %%%
	
\end{definition}

\begin{remark}
	Eine Überdeckungsrelation ist das gleiche wie ein {\em Abschlußoperator} $U\mapsto U^{\lhd}$ auf $\mathbb P(S)$, mit den folgenden Axiomen:
	\begin{itemize}
		\item {\em Reflexivität}: $U\subseteq U^\lhd$
		\item {\em Transitivität}: $U\subseteq V^\lhd \to U^\lhd \subseteq V^\lhd$ % endspricht CUT in kalkülen -> schlecht
	\end{itemize}
	Korrespondenz $\lhd \leftrightsquigarrow \_^{\lhd}$:
	Zu $\lhd$ definiere $U^\lhd= \{a\in S: a\lhd U\}$.
	$a\lhd U \leftrightsquigarrow a\in U^\lhd$.

	Alternatives Axiomensystem:
	\begin{itemize}
		\item Reflexivität: wie oben.
		\item Monotonie: $U\subseteq V \to U^\lhd \subseteq V^\lhd$
		\item Idempotenz: $U^{\lhd\lhd} \subseteq U^{\lhd}$. (mit Refl. sogar $=$)
	\end{itemize}
	[R+T $\to$ M; T$\to$I; M+I$\to$T]
\end{remark}

\begin{definition}
	Eine Überdeckungsrelation $\lhd$ heißt
	\begin{itemize}
		\item {\em unitär} oder {\em Schottsch}, wenn aus $a\lhd U$ folgt: $\exists_{b\in U}(a\lhd\{b\})$.
		\item {\em finitär} oder {\em Stonesch}, wenn aus $a\lhd U$ folgt: $\exists_{U_0\subseteq U} (U_0 \text{ endlich} \land a\lhd U_0)$.
	\end{itemize}
	Eine finitäre Überdeckungsrelation $\lhd$ heißt {\em Abhängigkeitsrelation}, wenn $\lhd$ die {\em Abhängigkeitseigenschaft} hat, d.h.
	wenn für alle $a,b\in S$, $U\subseteq S$ gilt: 
	\[ a\lhd U\cup\{b\}\to a\lhd U\lor b\lhd U\cup\{a\}\]
	Ein $U\subseteq S$ heißt {\em ($\lhd$-)abhängig}, wenn $\exists_{b\in U}(b\lhd U-\{b\})$.
	
	$U$ heißt {\em ($\lhd$-)unabhängig}, wenn  $\forall_{b\in U}(b\nlhd U-\{b\})$.
\end{definition}

\begin{remark}
	$U$ abhängig $\to$ $U\neq\emptyset$. Außerdem ist $\emptyset$ unabhängig.
\end{remark}

\begin{example}
	\begin{enumerate}[(a)]
		\item $S$ Menge; $a\lhd U \equiv a\in U$, d.h. $U^\lhd=U$; Dann $\lhd$ unitär und jedes $U\subseteq S$ ist unabhängig.
		\item $S$ Vektorraum; $U^\lhd=(U)$ der von $U$ erzeugte Untervektorraum. $\lhd$ finitär; ``(un)abhängig'' ist ``linear (un)abhängig'' (!)
		\item $R\subseteq S$ komm. Ringe; für $U\subseteq S$ sei $R[U]$ die {\em Ringadjunktion} von $U$ an $R$ in $S$, d.h.
			\[ R[U]= \bigcup_{n\ge 0} \{ f(u_1,\ldots,u_n)\colon f\in R[X_1,\ldots,X_n]; u_1,\ldots, u_n \in U\} \]
			$U^\lhd = \bar{ R[U] }^S$ ganzer Abschluß von $R[U]$ in $S$, d.h. $a\lhd U\iff a^n = r_1a^{n-1}+\ldots+r_{n-1} a + r_n$ für geeignete $n\ge 1$; $r_1,\ldots,r_n\in R[U]$.
			
			Dann: $\lhd$ finitäre Überdeckungsrelation.
			$R, S$ Körper, $R(U)$ statt $R[U]\to \lhd$ Abhängigkeitsrelation und ``$\lhd$-(un)abhängigkeit'' ist ``algebraisch (un)abhängig''.
			Siehe B.L. van der Warden, (Moderne) Algebra I, §64.
	\end{enumerate}
\end{example}

\begin{definition}
	Nun sei $\lhd$ wieder eine allgemeine Abhängigkeitsrelation.
	$U$ {\em erzeugt} $S$, wenn $S\lhd U$, d.h. $\forall_{b\in S}(b\lhd U)$.
	$U$ {\em Basis}, wenn $U$ unabhängig, und $U$ erzeugt $S$.
	
	In den Beispielen (b) und (c) ist eine Basis eine Vektorraumbasis bzw. eine Tanszendenzbasis.
	$U$ ist {\em maximal unabhängig} genau dann, wenn jedes $V\subseteq S$ mit $V\supsetneq U$ abhängig ist (*), sowie $U$ unabhängig ist.
\end{definition}

\begin{lemma}
	$U$ maximal unabhängig $\iff$ $U$ Basis
	\begin{proof}
		``$\Rightarrow$`` (gilt i.a. nicht für $\lhd$ nur Überdeckungsrelation):
		Zeige: (*) $\land$ $S\ntriangleleft U \to U \text{ abhängig}$.
		Nehme $b\in S$ mit $b\nlhd U$.
		Speziell $b\notin U$, d.h. $U\subsetneq U\cup \{b\}$.
		Nach (*) ist $U\cup \{b\}$ abhängig, d.h. es gibt $a\in U\cup \{b\}$ mit $a\lhd (U\cup \{b\})-\{a\}$.
		Sofort folgt $a\neq b$ [$a=b\to b\lhd U$ $\lightning$].
		Also $a\lhd (U-\{a\})\cup\{b\}$.
		Nach Abhängigkeitseigenschaft ist dann $a\lhd U-\{a\}$, damit $U$ abhängig, da ja $a\in U$ wegen $a\neq b$, 
		oder $b\lhd (U-\{a\})\cup\{a\}$, d.h. $b\lhd U$. $\lightning$.

		``$\Leftarrow$'': Nur zu Zeigen: (*).
		Ist $V\supsetneq U$, etwa $b\in V-U$, so ist $U-\{b\}=U$ und $b\lhd U$, also $b\lhd U-\{b\}$ und damit $b\lhd V-\{b\}$, also $V$ abhängig.
	\end{proof}
\end{lemma}

% Mail an pschust@maths.leeds.ac.uk Betreff: transfinit

\begin{theorem}
	Zu jedem unabhängigen $U\subseteq S$ gibt es eine Basis $W$ mit $W\supseteq U$.
	\begin{proof}[Beweis mit ZL]
		Verwende obiges Lemma.
		Es sei $G=E\cap F$, wobei $E=\{V\subseteq S\colon V\supseteq U\}$ und 
		$F = \{V\subseteq S\colon V \text{unabhängig}\}$ ein maximales Element $W$ von $G$ ist die gewünschte Basis.
		Zu Zeigen: $G\neq \emptyset$ und $G$ dcpo.
		Nun ist $U\in G$.
		Ist $D\subseteq G$ gerichtet, so ist $\bigcup D\in E$, da $D\neq \emptyset$, sowie $\bigcup D\in F$, da $F$ abgeschlossen, da $F$ von endl. Charakter (Lemma oben).
	\end{proof}
\end{theorem}

\begin{gelaber}
	{\bf Typische Anwendung} $V$ Vektorraum, $x\in V$. 
	Dann gilt: $\forall_{\varphi \in V^*} ( \varphi(x)=0)\to x=0$.
	\begin{proof}[Indirekter Beweis, mit ZL]
		Wäre $x\neq 0$, so wäre $U=\{x\}$ unabhängig, also gäbe es (Satz) eine Basis $W$ von $V$ mit $W\supseteq U$; d.h. $x\in W$.
		Definiere $\varphi \in V*$ durch $\varphi(x)=1$ und $\varphi\upharpoonright W-\{x\}=0$.
		Dann $\varphi(x)\neq 0$.$\lightning$.
	\end{proof}
\end{gelaber}

%20. Juni, 1. Hälfte

\marginpar{\tiny{20.06.2013}}

$V$ Vektorraum über $K$, $\bigcap\limits_{\varphi\in V^*} \ker(\varphi)=\{0\}$ (*)

{\bf Direkter Beweis von (*) mit OI:} $U, W, \ldots$: Untervektorräume [OI: $X$ dcpo, $G\subseteq X$ offen und progressiv $\Rightarrow G=X$]. $E, F,
\ldots$: endlich Erzeugte Untervektorräume. $X$ bestehe aus allen Untervektorräumen. $X$ partielle Ordnung mit $\subseteq$, sogar dcpo mit $\bigvee
D=\bigcup D$ für $D\subseteq X$ gerichtet.

{\bf Beachte:} $\bigcup D\in X$, da $D\subseteq X$ gerichtet. [Nun: zu $x,y\in \bigcup D$, das heißt $x\in U\in D$, $y\in W\in D$, also gibt es $Z\in
  D$ mit $U\subseteq Z, W\subseteq Z$. Dafür: $x, y\in Z$, aber $x+y\in Z$, weshalb $x+y\in\bigcup D$. Ferner ist $\lambda\cdot x\in U$, also
  $\lambda\cdot x\in\bigcup D$. Warum $0\in\bigcup D$? Nun: $D\neq\emptyset$, etwa $T\in D$, wofür $0\in T$, also $0\in\bigcup D$.

(*) kann man schreiben wie folgt: Für $x\in V$ gilt: $x\in\bigcap\limits_{\varphi\in U^*}\ker(\varphi) \Rightarrow \underbrace{\forall U\in X: x\in
    U}_{\mbox{d.h. }x=0}$.

Es sei $G=\{U\in X\mid x\in U\}$. Zu zeigen: für festes $x\in V$ gilt $x\in\bigcap\limits_{\varphi\in V^*}\ker(\varphi)\Rightarrow G=X$. Nach OI zu
zeigen: (1) $G$ offen, (2) $G$ progressiv.

(1) zu zeigen: $\bigcup D\in G\Rightarrow D\cap G\neq\emptyset$. Nun: $\bigcup D\in G$ heißt $x\in\bigcup D$, das heißt es gibt $W\in D$ mit $x\in W,
W\in G$.

(2) sei $U\in X$ mit $W\in G$ für jedes $W\in X$ mit $W\supsetneq U$. Zeige $U\in G$.

Fall 1: $U=V$. Dann $x\in U$, das heißt $U\in G$.

Fall 2: $U\subsetneq V$. Nehme $y\in V\backslash U$. Dann $U(y)=U+Ky$.

Fall 2.1: $U(y)=V$. Dann sogar $U\oplus K(y)=V$ [$U\ni\lambda\cdot y\Rightarrow\lambda=0$, da $\lambda\neq 0\Rightarrow y\in U$ $\lightning$]. Definiere $\varphi\in V^*$ durch $\varphi\upharpoonright U=0$,
$\varphi(y)=1$. Nach Voraussetzung ist $x\in\ker\varphi=U$, also $U\in G$.

Fall 2.2: $U(y)\subsetneq V$. Nehme $z\in V\backslash U(y)$. Nun
$U\subsetneq U(y)$ und $U\subsetneq U(z)$, also (nach Ind.) $U(y)\in G$ und $U(z)\in G$, das heißt $x\in U(y)$ und $x\in U(z)$, also $x\in U$ (nach
Lemma unten), das heißt $U\in G$.

\begin{lemma}
  $z\not\in U(y)\Rightarrow U(y)\cap U(z)\subseteq U$
  \begin{proof}
    Ist $x\in U(y)\cap U(z)$, das heißt $x=u+\lambda y\land x=v+\mu z$ für $u,v\in U$, so ist $\mu = 0$ [aus $\mu\neq 0$ folgte
      $z=\mu^{-1}(x-v)=\mu^{-1}(\underbrace{u-v}_{u}+\lambda v)\in U(y)$ $\lightning$], also $x=v\in U$.
  \end{proof}
\end{lemma}

Beweis mit ZL des Satzes, dass für jede Abhängigkeitsrelation $\lhd$ auf einer Menge $S$ für $U, V\subseteq S$ gilt: $U, V$ unabhängig und $U, V$
äquivalent (das heißt $U\lhd V\land V\lhd U$) impliziert $U, V$ gleichmächtig (das heißt es gibt Bijektion $U\rightarrow V$) (§).

Verwende: Satz von Cantor-Bernstein-Schröder: Sind $U, V$ Mengen und gibt es injektive Abbildungen $U\rightarrow V$ und $V\rightarrow U$, so sind $U$
und $V$ gleichmächtig. Da (§) symmetrisch ist in $U$ und $V$, reicht es zu zeigen: es gibt eine bijektive Abbildung $U\rightarrow V^*$ mit
$V^*\subseteq V$.

Dazu sei $W=U\cap V$, eventuell $W=\emptyset$. Definiere $X=\{\varphi':U'\rightarrow V'\mid \varphi\mbox{ bijektiv}, W\subseteq U'\subseteq U,
W\subseteq V'\subseteq V, V'\cup(U\backslash U')\mbox{ unabhängig}\}$. Wegen $\operatorname{id}_W\in X$ ist $X\neq\emptyset$.

Ordne $X$ durch $\varphi'\le\varphi''\equiv \varphi'\subseteq\varphi''$, das heißt, dass $\varphi''$ fortsetzung von $\varphi'$ ist, das heißt mit
$\varphi':U'\rightarrow V'$ und $\varphi'':U''\rightarrow V''$ ist $U'\subseteq U''$, $V'\subseteq V''$ und $\varphi''\upharpoonright U'\rightarrow
V'$ ist gleich $\varphi'$. Nun ist $X$ ein dcpo. Ist $D\subseteq X$ gerichtet, so ist $\bar{\varphi}=\bigcup D$ Element von $X$ (!); haben: $\bar{\varphi}=\bigvee D$.

Zu (!): $\bar{\varphi}:\bar{U}\rightarrow\bar{V}$ mit $\bar{U}=\bigcup\{\dom(\varphi')\mid \varphi'\in D\}$; analog
$\bar{V}=\bigcup\{\codom(\varphi')\mid \varphi'\in D\}$; klar: $\bar{\varphi}$ bijektiv.

%20. Juni, 2. Hälfte
Zeige: $\bar{V}\cup(U\backslash\bar{U})$ ist unabhängig. Erinnerung: ``Unabhängig'' von endl. Char. Ist $F\subseteq \bar{V}\cap(U\backslash\bar{U})$,
$F$ endlich, so gibt es (da $D$ gerichtet) $\varphi':U'\rightarrow V'$ mit $\varphi'\in D$ und $F\subseteq \underbrace{V'\cup(U\backslash
  U')}_{\mbox{unabh.}}$, also $F$ unabhängig. Nach ZL hat $X$ ein maximales Element $\varphi^*:U^*\rightarrow V^*$. Noch zu zeigen: $U^*=U$. Annahme
$U^*\subsetneq U$. Nehme $u\in U\backslash U^*$. Nun: $R^*=V^*\cup(U\backslash U^*)$. $R^*$ unabängig (da $\varphi^*\in X$), also $V\nlhd
R^*\backslash\{u\}$ [sonst $u\lhd R^*\backslash\{u\}$ (da $U\lhd V$; Transitivität)], etwa $v\in V$ mit $v\nlhd R^*\backslash\{u\}$. Betrachte
$R^{**})=(V^*\cup\{v\})\cup(U\backslash(U^*\cap\{u\}))=(R^*\backslash\{u\})\cup\{v\}$. $R^{**}$ ist unabhängig, das heißt $\forall_{r\in
  R^{**}}(r\nlhd R^{**}\backslash\{r\})$ [sonst, etwa $r\in R^\{**\}$ mit $r\lhd R^{**}\backslash\{r\}$, dann $r\neq v$ ($r=v: v\lhd
  R^{**}\backslash\{v\}=R^*\backslash\{u\}$ $\lightning$), also $r\in R^*\backslash\{u\}$, aber $r\lhd R^{**}\backslash\{r\}$ heißt damit $r\lhd
  (R^*\backslash\{u,r\})\cup\{v\}$, weshalb $r\lhd R^*\backslash\{r\}$ $\lightning$ ($R^*$ unabhängig) oder
  $v\lhd(R^*\backslash\{u,r\})\cup\{r\}=R^*\backslash\{u\}$ $\lightning$]. Also wäre $\varphi^{**}:U^*\cup\{u\}\rightarrow V^*\cup\{v\}$ mit
$\varphi^{**}(x) = \left\{ \begin{array}{rl} \varphi^*(x),& x\in U^* \\ v,& x = u \end{array}\right.$ in $X$ mit $\varphi^* < \varphi^{**}$ im
Widerspruch zur Maximalität von $\varphi^*$.

\section{Maximale Ideale}

\begin{gelaber}
  {\bf Erinnerung:} Ring $(R,+,\cdot,0,1)$ mit $(R,+,\cdot)$ abelsche Gruppe, $(R,\cdot,1)$ Halbgruppe, $x(y+z)=xy+xz$, $(x+y)z=xz+yz$. Stets sei $R$
  kommutativ, das heißt $xy=yx$ gilt in $R$. $x$ heißt {\bf Einheit} von $R$ oder {\bf invertierbar}, wenn es ein $y$ gibt mit $xy=1$. $R^*$ sei die
  Gruppe der Einheiten von $R$, $R^*$ abelsch. Ein Ring $R$ ist ein Körper, wenn $1\neq 0$ und $R^*=R\backslash\{0\}$. {\bf $R$-Modul}: $M$ ist eine
  abelsche Gruppe $(M,+,0)$ mit $R\times M\rightarrow M$, $(r,x)\mapsto rx$, sodass $r(sx)=(rs)x$, $1x=x$, $r(x+y)=rx+ry$, $(r+s)x=rx+sx$. {\bf
    $U\subseteq M$ ($R$-)Untermodul}: $U$ Untergruppe von $(M,+,0)$ derart dass gilt: $r\in R \land x\in U\Rightarrow rx\in U$. $Rx=\{rx\mid r\in R\}$,
  $U+V=\{u+v\mid u\in U, v\in V\}$. $(X_1,\ldots,X_n)=Rx_1+\ldots+Rx_n$. Jede abelsche Gruppe ist ein $\mathbb{Z}$-Modul via $n\cdot
  x=\underbrace{x\cdot x}_{n\mbox{-mal}}$, $(-n)x=-(nx)$, für $n\ge 0$. Ein {\bf Ideal} von $R$ ist ein $R$-Untermodul von $R$. Ein Ideal $M$ von $R$
  heißt {\bf maximales Ideal}, wenn $M\neq R$ und für jedes Ideal $I$ von $R$ gilt: $M\subseteq I \Rightarrow M=I \vee I=R$. Das heißt, $M$ ist
  maximal unter den Idealen $I$ mit $I\neq R$.
\end{gelaber}

\begin{theorem}
  Es sei $R$ ein Ring. Zu jedem Ideal $I$ von $R$ mit $I\neq R$ gibt es ein maximales Ideal $M$ von $R$ mit $M\supseteq I$.
\end{theorem}

\begin{corollary}
  Jeder Ring, in dem $1\neq 0$, hat ein maximales Ideal. [Satz mit $I=\{0\}$]
\end{corollary}

\begin{proof}[Beweis Satz mit ZL]
$X=\{F\subseteq R:F\mbox{ Ideal}, F\neq R, F\supseteq I\}$ ist $\neq\emptyset$ (da $I\in X$) und ein dcpo, denn ist $D\subseteq X$ gerichtet, so ist
  $\bigcup D$ ein Ideal (!) mit $\bigcup D\supseteq I$ (!) und $\bigcup D\neq R$ [sonst wäre $1\in \bigcup D$, etwa $1\in F\in D$, also $F=R$ nach
    Bemerkung unten $\lightning$]. Also: $\bigcup D\in X$. Nach ZL hat $X$ ein maximales Element, das heißt $R$ hat ein maximales Ideal $\supseteq I$.
\end{proof}

\begin{gelaber}
  {\bf Bemerkung:} Für jedes Ideal $F$ eines Rings $R$ sind äquivalent:
  \begin{itemize}
    \item[i.] $F=R$
    \item[ii.] $1\in F$
    \item[iii.] $R^*\cap F\neq\emptyset$
  \end{itemize}
  \begin{proof}
    $i.\Rightarrow ii.\Rightarrow iii.$ trivial. $iii.\Rightarrow i.$: Ist etwa $r\in R^*\cap F$, so gibt es $s\in R$ mit $rs=1$; für $t\in F$ gilt
    dann $t=tsr\in F$ weil $r\in F, sr=1$.
  \end{proof}
\end{gelaber}

\begin{gelaber}
  {\bf Variante von ZL}

  ZL': Ist $X$ ein dcpo, so gibt es zu jedem $x\in X$ ein maximales Element $y\in X$ mit $y\ge x$.

  ZL' $\Rightarrow$ ZL: trivial.

  ZL $\Rightarrow$ ZL': Es sei $x\in X$. Mit $X$ dcpo ist auch $\uparrow x = \{z\in X\mid z\ge x\}$ ein dcpo mit $\uparrow x \neq\emptyset$ (da
  $x\in\uparrow x$).
\end{gelaber}

%% 4. Juli 2013

Allgemeiner, aber analog: $R$ kommutativer Ring, $V$ ein $R$-Modul.

\begin{theorem}
  Zu jedem Untermodul $U$ von $V$ gibt es einen maximalen Untermodul $M$ von $V$ mit $M\supseteq U$.
\end{theorem}

\begin{corollary}
  Ist $V\neq 0$, er hat $V$ einen maximalen Untermodul.
\end{corollary}

Aus Korollar folgt Satz: Gehe von $V$ zu $V/U$. Ebenso für Ideale. ($V/U\neq 0\Leftrightarrow V\neq U$.) Ebenso für Ideale.

Zurück zum kommutativen Ring $R$. $\Max(R)$ ist die Menge der maximalen Ideale von $R$.

\begin{remark}
  Es gilt:
  \begin{itemize}
    \item $\bigcup\limits_{M\in\Max(R)} M =_{(1)} R\backslash R^\times$
    \item $\bigcap\limits_{M\in\Max(R)} (R\backslash M) =_{(2)} R^\times$
  \end{itemize}
  ZL für $\supseteq$ in (1), $\subseteq$ in (2)
\end{remark}

\begin{definition}
  $\Jac(R) = \{x\in R\mid \forall_{y\in R} (1-xy\in R^\times)\}$ {\bf Jacobson-Radikal} von $R$
\end{definition}

\begin{theorem}
  $\bigcap\Max(R)=\Jac(R)$
\end{theorem}

Es gilt $1\in\Jac(R)\Rightarrow 1=0$ [$x=1,y=1\Rightarrow 0=1-xy\in R^\times$, das heißt $1=0z=0$ für ein $z\in R$]

\begin{proof}[Beweis Satz mit ZL für $\subseteq$]
  $\phantom{a}$

  \begin{itemize}
    \item $\subseteq$: Es sei $x\in M$ für alle $M\in\Max(R)$. Angenommen, es gäbe $y\in R$ mit $1-xy\not\in R^\times$. Denn $I=(1-xy)\subsetneq R$,
      also gibt es $M\in\Max(R)$ mit $M\supseteq I$, wofür $x\not\in M$ $\lightning$ [$x\in M\Rightarrow 1=1-\underbrace{1-xy}_{\in
          M}+\underbrace{xy}_{\in M}\in M$, also $M=R$ $\lightning$].
    \item $\supseteq$: Es sei $x\in\Jac(R)$ und $M\in\Max(R)$. Annahme: $x\not\in M$, das heißt $I=M+(x)\supsetneq M$. Da $M\in\Max(R)$, ist $I=R$,
      das heißt $1\in I$, das heißt $1=z+xy$ für ein $z\in M$ und ein $y\in R$, also $M\ni z=1-xy\in R^\times$, $x\in\Jac(R)$, also $M=R$
      $\lightning$.
  \end{itemize}
  Insgesamt $x\in M$.
\end{proof}

\begin{gelaber}
  ZL: $X$ dcpo, $F\subseteq X$ abgeschlossen, $F\neq\emptyset$ $\Rightarrow$ $F$ hat maximales element.
\end{gelaber}

\begin{gelaber}
  OI: $X$ dcpo, $G\subseteq X$ offen, $G$ progressiv $\Rightarrow$ $G=X$.

  $G$ offen: $D\subseteq X$ gerichtet, $\bigvee D\in G$ $\Rightarrow$
  $\exists_{x\in D}(x\in G)$.

  $G$ progressiv: $\forall_x . \forall_{y>x}(y\in G) \Rightarrow x\in G$
\end{gelaber}

$\subseteq$ im obigen Satz folgt aus

\begin{theorem}
  $a\in\bigcap\Max(R)\Rightarrow 1-a\in R^\times$ (nehme $xy$ als $a$)
\end{theorem}

Beweis mit OI und folgendem Lemma.

\begin{lemma}
  Es sei $X$ ein dcpo; $F,G\subseteq X$.
  \begin{itemize}
    \item[a.] $F$ monoton $\Rightarrow$ $F$ abgeschlossen.
    \item[b.] $G$ antimonoton $\Rightarrow$ $G$ offen.
  \end{itemize}
  wobei $F$ {\bf monoton}, wenn $F\ni x\le y\Rightarrow y\in F$, $(x\le y \Rightarrow (x\in F\Rightarrow y\in F))$, $G$ {\bf antimonoton}, wenn $x\le
  y\in G\Rightarrow x\in G$, $(x\le y\Rightarrow (x\in G\Leftarrow y\in G))$
\end{lemma}

\begin{proof}[Beweis Lemma]
  $\phantom{a}$
  \begin{itemize}
    \item[a.] Es sei $D\subseteq X$, $D$ gerichtet, speziell $D\neq\emptyset$, etwa $z\in D$. Ist $D\subseteq F$, so ist $z\in F$, aber $z\le\bigvee
      D$, also $\bigvee D\in F$, da $F$ monoton.
    \item[b.] $D\subseteq X$ gerichtet, $\bigvee D\in G$, $z\in D$, $z\le\bigvee D$ $\Rightarrow$ $z\in G$ ($G$ antimonoton)
  \end{itemize}
\end{proof}

\begin{proof}[Beweis Satz]
  Es sei $a\in\bigcap\Max(R)$. $X=\{\mbox{Ideale }I\neq R\}$ ist ein dcpo. $G=\{I\in X\mid 1-a\not\in I\}$ antimonoton
  $\underbrace{\Rightarrow}_{\mbox{\tiny Lemma}}$ $G$ offen. $G$ progressiv: Dazu sei $I\in X$ derart, dass $J\in G$ für alle $J\in X$ mit
  $J\supsetneq I$, zu zeigen $I\in G$. Fall 1: $I\in\Max(R)$. Dann $a\in I$, also $1-a\not\in I$ [$1-a\in I\Rightarrow 1=\underbrace{1-a}_{\in
      I}+\underbrace{a}_{\in I}$, das heißt $I=R$ $\lightning$]. Somit $I\in G$. Fall 2: $I\not\in\Max(R)$, das heißt es gibt $J\in X$ mit
  $J\supsetneq I$. Nach Induktion: $J\in G$, also $I\in G$ ($G$ antimonoton). Insgesamt $G=X$ mit OI, also: $1-a\in R^\times$.
\end{proof}

{\bf Beweismuster.} (*) $X$ dcpo, $G\subseteq X$ antimonoton, $\Max(X)\subseteq G\Rightarrow G=X$, wobei $\Max(X)=\{\mbox{max. elem. von }X\}$.

(*) Folgt aus OI wie in Beweis oben. Genügt für den Satz.

{\bf Anwendung.} $X$ topologischer Raum, $A$ Unteralgebra von ${\cal C}(X,\mathbb{C})$. Für jedes $x\in X$ ist $M(x)\in\Max(A)$, wobei $M(x)=\{f\in
A\mid f(x)=0\}$. [$M(x)=\Ker(\alpha_x)$ mit $\alpha_x:A\twoheadrightarrow\mathbb{C},f\mapsto f(x)\Rightarrow A/M(x)\cong\mathbb{C}$ Körper.]

\begin{definition}
  $A$ {\bf Gelfand-Neumark-Algebra (GNA)} wenn $\Max(A)=\{M(x):x\in X\}$.
\end{definition}

\begin{theorem}
  Ist $A$ ein GNA, und $f\in A$, mit $\forall_{x\in X}(f(x)\neq 0)$, so ist $\underbrace{f\in A^\times}_{\frac{1}{f}\in A}$ $\Rightarrow$
  $\frac{1}{f}\in{\cal C}(X,\mathbb{C})$.
\end{theorem}

\begin{proof}[Beweis mit (*), also OI]
  Es sei $f\in A$ mit $\forall_{x\in X}(f(x)\neq 0)$. $\{\mbox{echte Ideale von} A\}=:E$ dcpo. Zu zeigen: $\forall_{I\in E}(f\not\in I)$, das heißt
  $E=G$ mit $G=\{I\in E\mid f\not\in i\}$ antimonoton. Nun: $\Max(A)\subseteq G$, denn ist $M\in\Max(A)$, so ist $M=M(x)$ für ein $x\in X$, da $A$
  eine GNA, also $f\not\in M$, das heißt $M\in G$, denn aus $f\in M=M(x)$ folgte $f(x)=0$ $\lightning$.
\end{proof}

{\bf Beispiel.} Wiener-Algebra $A=\{\sum\limits_{n\in\mathbb{Z}} a_ne^{int}\mid\sum\limits_{n\in\mathbb{Z}}|a_n|<\infty\}$.

{\bf Anwendung.} Der algebraische Abschluss eines Körpers $K$. Ein Körper $\bar{K}$ heißt {\bf algebraischer Abschluss} von $K$, wenn
$\bar{K}\supseteq K$, $\bar{K}$ algebraisch über $K$ und $\bar{K}$ algebraisch abgeschlossen, das heißt zu jedem Polynom
$f\in\bar{K}[T]\backslash\bar{K}$ gibt es $x\in\bar{K}$ mit $f(x)=0$. Zum Beispiel ist
$\bar{\mathbb{Q}}=\{x\in\mathbb{C}\mid\exists_{g\in\mathbb{Q}[T]\backslash\mathbb{Q}}(g(x)=0)\}$ ein algebraischer Abschluss von $\mathbb{Q}$ mit
$\bar{\mathbb{Q}}\subsetneq\mathbb{C}$, zum Beispiel $\pi, e\not\in\mathbb{C}\backslash\bar{\mathbb{Q}}$.

\begin{theorem}[Steinitz 1910; Beweis nach Artin]
  Jeder Körper $K$ hat einen algebraischen Abschluss $\bar{K}$.
\end{theorem}

\begin{lemma}[Kronecker]
  Zu $f_1,\ldots,f_n\in K[T]\backslash K$ gibt es Körper $L\supseteq K$, sodass jedes $f_i$ in $L$ eine Nullstelle hat. [$n=1$: oBdA $f$ irreduzibel,
    $L=K[T]/(f), x=\bar{T}$.]
\end{lemma}

\begin{proof}[Beweis Satz]
  Es sei $\Lambda=K[T]\backslash K$; für jedes $f\in\Lambda$ sei $X_f$ eine Unbestimmte. Sei $P=K[\{X_f\mid f\in\Lambda\}]$, und $I=(\{f(X_f)\mid
  f\in\Lambda\})$ als Ideal von $P$.

  Behauptung: $1\not\in I$. Beweis: $1\in I\rightarrow 1=_{(*)}\sum\limits_{i=1}^r g_i f_i(X_{f_i})$, $g_i\in P$. Nach Lemma gibt es Körper
  $L\supseteq K$ und $x_1,\ldots,x_r\in L$ mit $f_i(x_i)=0$. Setze $x_i$ ein für $X_{f_i}$ ein (*). Dann $1=\sum\tilde{g_i}f_i=0$ in $L$ $\lightning$.

  Sei $M$ maximales Ideal über $I$. $E_1=P/M$ ist Körper, $E_1\supseteq K$ wegen $P\supseteq K$. Wegen $P=K[\{X_f\mid f\in\Lambda\}]$ ist $E_1=K[\{x_f:f\in\Lambda\}]$ mit $x_f=X_f+M\in
  E_1$. Wegen $M\supseteq I$ ist $f(x_f) = f(X_f)\mod M = 0\mod M$ für $f\in\Lambda$. Nun wiederhole das Verfahren. Konstruiere Körper $K=E_0\subseteq
  E_1\subseteq E_2\subseteq \ldots$ wie folgt: Zu $E_n$ konstruiere $E_{n+1}$ ``wie $E_1$ zu $K$''. Damit ist $E_{n+1}|E_n$ algebraisch, und jedes
  $h\in E_n[T]\backslash E_n$ hat Nullstelle in $E_{n+1}$. Insgesamt ist $\bar{K}=\bigcup\limits_{n\ge 0}E_n$ ein Körper: Alle $E_n|K$ algebraisch,
  also $\bar{K}|K$ algebraisch. Ist $f\in\bar{K}[T]\backslash\bar{K}$, etwa $f\in E_N[T]\backslash E_n$, so hat $f$ Nullstelle in
  $E_{N+1}\subseteq\bar{K}$.
\end{proof}

Mit ZL kann man zeigen: Sind $\bar{K},\tilde{K}$ algebraische Abschlüsse von $K$, so gibt es einen Isomorphismus $\varphi:\bar{K}\rightarrow\tilde{K}$
mit $\varphi\upharpoonright K=\id K$.

{\bf Übung.} (ZL) Es sei $S$ eine Menge. Zu jeder partiellen Ordnung $P\subseteq S\times S$ auf $S$ gibt es eine lineare Ordnung $Q\subseteq S\times
S$ auf $S$ mit $Q\supseteq P$.

{\bf Erinnerung.} Sind $A\subseteq B$ kommutative Ringe, $S\subseteq B$, so ist $A[S]=\bigcup\limits_{n\ge 0}\{f(s_1,\ldots,s_n):f\in
A[T_1,\ldots,T_n],s_1,\ldots,s_n\in S\}$ {\bf Ringadjunktion} von $S$ an $A$ in $B$.

{\bf Anwendung.} Hilbert'scher Nullstellensatz (HNS).

\begin{lemma}[Körpertheoretische Fassung des HNS]
  Sind $K\subseteq L$ Körper mit $L=K[a_1,\ldots,a_r]$, so ist $L|K$ algebraisch, sogar endlich.
\end{lemma}

\begin{theorem}[Schwache Form des HNS]
  Es sei $K$ ein Körper, $K$ algebraisch abgeschlossen. Für $f_1,\ldots,f_m\in K[T_1,\ldots,T_n]$ sei $I=(f_1,\ldots,f_m)$ in
  $R=K[T_1,\ldots,T_n]$. Ist $I\neq R$, so gibt es $t\in K^n$ mit $f_i(t)=0$ für alle $i\in\{1,\ldots,m\}$.
\end{theorem}

\begin{proof}[Beweis mit ZL]
  Wegen $I\neq R$ gibt es ein maximales Ideal $M$ mit $M\supseteq I$. Nun betrachte $M(t)=(T_1-t_1,\ldots,T_n-t_n)$ in $R$ für $t=(t_1,\ldots,t_n)\in
  K^n$, wofür $M(t)\in\Max(R)$ [$R \twoheadrightarrow_{\alpha_t} k, f\mapsto f(t)$ mit $\Ker(\alpha_t)\supseteq M(t)$, sogar $=$ (!).]

  Zu $M\in\Max(R)$ gibt es $t\in K^n$ mit $M=M(t)$ [siehe unten!]. Aus $M(t)=M\supseteq I\ni f_1,\ldots,f_m$ folgt $f_i(t)=0$ für alle $i\le m$.

  Es sei $M\in\Max(R)$. Dann ist $R/M$ Körper, $R/M=K[\bar{T_1},\ldots,\bar{T_n}]$ mit $\bar{T_i}=T_i+M$ in $R/M$, also $R/M$ algebraisch über $K$,
  somit $K=R/M$, das heißt $k\rightarrow R \twoheadrightarrow R/M$ ist Isomorphismus. Zu $\bar{T_i}\in R/M$ gibt es also $\bar{t_i}=\bar{T_i}$ in
  $R/M$. Dann $M(t)\subseteq M$, da $T_i-t_i\in M$, also $M(t)=M$ wegen $M(t)$ maximal.
\end{proof}

{\bf Alternativ mit Beweismuster.} $X=\{\mbox{Echte Ideale in }R\}$, $G=\{I\in X\mid\exists_{t\in K^n}(M(t)\supseteq I)\}$, $G$
antimonoton. $\Max(X)\subseteq G \underbrace{\Rightarrow}_{OI} G=X$.

\subsection{Primideale}

\marginpar{\tiny{11.07.2013}}

$p\in\mathbb{Z}\backslash\{0,-1,1\}$ heißt {\bf Primzahl}, wenn $p|ab\Rightarrow p|a \vee p|b$. Nun sei $R$ ein kommutativer Ring. Ideal $I$ von $R$
heißt {\bf Primideal}, wenn $1\not\in I$ und $ab\in I\Rightarrow a\in I \vee b\in I$. Zum Beispiel ist $(p)$ Primideal von $\mathbb{Z}$ genau dann,
wenn $p=0$ oder $p$ Primzahl. $S\subseteq R$ {\bf multiplikativ} oder {\bf Monoid}, wenn $1\in S$ und $a\in S\wedge b\in S\Rightarrow ab\in S$. Für
jedes Ideal $P$ von $R$: $P$ Primideal $\Leftrightarrow$ $R\backslash P$ Monoid. Weiter sind Monoide $\{1\},R^*,\left<a\right>=\{a^n\mid n\ge 0\},
S+I=\{s+x\mid s\in S,x\in I\}$ falls $S$ Monoid, $I$ Ideal: $(s+x)(t+y) = \underbrace{st}_{\in S}+\underbrace{sy+xt+xy}_{\in I}$.

$0$ Primideal $\Leftrightarrow$ $R$ {\bf Integritätsring}, das heißt $1\neq 0$ und $ab=0$ folgt $a=0\vee b=0$. Allgemein ist ein Ideal $P$ Primideal genau dann,
wenn $R/P$ Integritätsring. Ferner: Ein Ideal $M$ von $R$ ist maximal genau dann, wenn $R/M$ Körper.

{\bf Speziell} $\Max(R)\subseteq\Spec(R)$, wobei $\Spec(R)=\{P\underbrace{\subseteq}_{(*)} R\mid P\mbox{ Primideale}\}$ {\bf Maximal-/Primspektrum} von
$R$. (*) wird gleich gezeigt.

\begin{theorem}[Krullscher Primidealsatz]
  Es sei $R$ kommutativer Ring, $S\subseteq R$ Monoid, $I\subseteq R$ Ideal. Ist $I\cap S=\emptyset$, so gibt es ein $P\in\Spec(R)$ mit $P\supseteq
  I$ und $P\cap S=\emptyset$. Genauer gilt für $I\cap S=\emptyset$:
  \begin{itemize}
    \item[a.] $\Max(X)\neq\emptyset,X=\{F\subseteq R\mid F\mbox{ Ideal mit } F\supseteq I, F\cap S=\emptyset\}$.
    \item[b.] $\Max(X)\subseteq\Spec(R)$
  \end{itemize}
\end{theorem}

{\bf Spezialfälle:}
\begin{itemize}
  \item[1.] $S\in\{\{1\},R^*,\{1\}+I\}$ $\Rightarrow$ $X=\{\mbox{Ideale } F\supseteq I\mbox{ mit }F\subsetneq R\}$ $\Rightarrow$ $\Max(X)=\Max(R/I)$;
    a. bekannt, da dann $I\cap S=\emptyset$ heißt $I\subsetneq R$; b. für $I=0$: $\Max(R)\subseteq\Spec(R)$, das ist (*).
  \item[2.] $S=R\backslash P$, $P\in\Spec(R)$: $\Max(X)=\{P\}$
  \item[3.] $S=\left<a\right>,a\in R$: s.u.!
\end{itemize}

\begin{proof}[Beweis Satz mit ZL]
  Es sei $I\cap S=\emptyset$.
  \begin{itemize}
    \item[a.] $X\neq\emptyset$ wegen $I\in X$; $X\subseteq\mathbb{P}(R)$ abgeschlossen. $D\subseteq X$ gerichtet $\Rightarrow$ $\bigcup D\in X$. Wegen
      $\mathbb{P}(R)$ dcpo folgt mit ZL $\Max(X)\neq\emptyset$.
    \item[b.] Es sei $P\in\Max(X)$. Wegen $1\in S$ ist $1\not\in P$. Annahme: es gibt $a,b\in R\backslash P$ mit $ab\in P$. Für $G=P+(a)$, $H=P+(b)$
      gilt $G,H\supsetneq P$; $G,H\supseteq I$; also $G\between S, H\between S$\footnote{$A\between B :\Leftrightarrow A\cap B\neq\emptyset$}, etwa
      $x+ua\in S\ni y+vb$ mit $x,y\in P$; $u,v\in R$; also $S\ni(x+ua)(y+vb)=\underbrace{xy+xvb+uay}_{\in P}+\underbrace{uvab}_{\in P}$ $\lightning$.
  \end{itemize}
\end{proof}

\begin{definition}
  Für ein Ideal $I$ von $R$ heißt $\sqrt{I}=\{a\in R\mid\exists_{n\ge 0}(a^n\in I)\}$ {\bf Radikal} von $I$; $\sqrt(I)$ ist ein Ideal mit
  $\sqrt(I)\supseteq I$. $I$ heißt {\bf Radikalideal}, wenn $\sqrt{I}=I$; stets gilt: $\sqrt{\sqrt{I}}=\sqrt{I}$. $\sqrt{0}$ {\bf Nilradikal},
  kleinstes Radikalideal!
\end{definition}

Es gilt: $\sqrt{I}\ni a \Leftrightarrow I\between\left<a\right>$.

\begin{corollary}
  Für jedes Ideal $I$ gilt: $\sqrt{I}=\bigcap\{P\in\Spec(R)\mid P\supseteq I\}$. Speziell für $I=\{0\}$: $\sqrt{0}=\bigcap\Spec(R)$.
\end{corollary}

\begin{proof}{Korollar mit Satz}
  \begin{itemize}
    \item $\subseteq$: $a\in\sqrt{I}$, etwa $a^n\in I$, dann $a^n\in P$, also $a\in P$ für alle Primideale $P\supseteq I$. [hier: Primideale sind
      Radikalideale!]
    \item $\supseteq$: Ist $a\not\in\sqrt{I}$, das heißt $I\cap\left<a\right>=\emptyset$, so gilt nach Satz dass es ein $P\in\Spec(R)$ gibt mit
      $P\supseteq I$ und $P\cap\left<a\right>=\emptyset$, das heißt $a\not\in P$.
  \end{itemize}
\end{proof}

\subsection{Ein universeller Satz von Krull-Lindenbaum}

\begin{definition}
  Es sei $X$ ein dcpo, in dem zu $x,y\in X$ die größte untere Schranke $x\wedge y\in X$ existiert, charakterisiert durch $\forall_{z\in X}(z\le
  x\wedge y\Leftrightarrow z\le x\wedge z\le y)$. Ein $T\subseteq X$ heißt {\bf $\wedge$-abgeschlossen}, wenn aus $x\in T$ und $y\in T$ folgt $x\wedge
  y\in T$. Ein $x\in X$ heißt {\bf irreduzibel}, wenn aus $x=y\wedge z$ folgt $x=y$ oder $x=z$ ($y,z\in X$). Damit: $x\in X$ {\bf reduzibel}, wenn es
  $x,z\in X$ gibt mit $x=y\wedge z$ und $x<y$ und $x<z$.
\end{definition}

\begin{theorem}[Beweismuster]
  Ist $X$ wie oben, und $T\subseteq X$ offen und $\wedge$-abgeschlossen, so gilt: Gehört jedes irreduzible $x\in X$ zu $T$, so ist $T=X$. Kurz:
  $\Irr(X)\subseteq T\Rightarrow T=X$\footnote{$\Irr(X)$ = irreduzible Elemente von $X$}.
\end{theorem}

\begin{proof}[Beweis mit OI]
  Zu zeigen: $T$ progressiv. Dazu sei $x\in X$ mit $\forall_{y>x}(y\in T)$. Zu zeigen: $x\in T$. Fall 1: $x$ irreduzibel, dann nach voraussetzung
  $x\in T$. Fall 2: $x$ reduzibel, etwa $x=y\wedge z$ mit $x<y,x<z$. Nach Induktionsvoraussetzung: $y,z\in T$. Da $T$ $\wedge$-abgeschlossen, folgt
  $x\in T$.
\end{proof}

\begin{theorem}[Beweismuster]
  $X$ dcpo mit $\wedge$; $T\subseteq X$ offen, $\wedge$-abgeschlossen, $\forall_{x\in X}(\Irr(x)\Rightarrow x\in T)\Rightarrow\forall_{x\in X}(x\in
  T)$.
\end{theorem}

\begin{definition}
Nun sei $\lhd$ eine Überdeckungsrelation auf Menge $S$, $U^\lhd=\{a\in S\mid a \lhd U\}$; Axiome: Reflexivität $U\subseteq U^\lhd$; Transitivität
$U\subseteq V^\lhd\Rightarrow U^\lhd\subseteq V^\lhd$. $U\subseteq S$ {\bf saturiert} oder ein {\bf Ideal}, wenn $U=U^\lhd$.
\end{definition}

$\Sat(\lhd)=\{U\subseteq S\mid U=U^\lhd\}$ ist eine gegenüber $\cap$ und, falls $\lhd$ finitär, gerichteten Vereinigungen abgeschlossene Teilmenge von
$\mathbb{P}(S)$. In der Tat gilt: $I,J\in\Sat(\lhd)\Rightarrow I\cap J\in\Sat(\lhd)$ [zu zeigen: $a\lhd I\cap J\Rightarrow a\in I\cap J$. Nun: $a\lhd
  I\cap J\Rightarrow a\lhd I\Rightarrow_{I=I^\lhd} a\in I$; genauso für $J$.] Ist ferner $D$ eine gerichtete Teilmenge von $\Sat(\lhd)$, so ist
$\bigcup D\in\Sat(\lhd)$, denn: zu zeigen: $a\lhd\bigcup D\Rightarrow a\in\bigcup D$. Aber aus $a\lhd\bigcup D$ folgt (da $\lhd$ finitär) $a\lhd D_0$
für endliches $D_0\subseteq D$, also $a\lhd F$ für ein $F\in D$ (obere Schranke von $D_0$), das heißt $a\in F$, also $a\in\bigcup D$.

Für $F\in\Sat(\lhd)$ ist $\uparrow F=\{G\in\Sat(\lhd)\mid G\supseteq F\}$ eine gegenüber $\cap$ und gerichteten Vereinigungen abgeschlossene Teilmenge
von $\Sat(\lhd)$, denn ist $D\subseteq\uparrow F$ gerichtet, so ist $D\neq\emptyset$, etwa $G\in D$, wofür $\bigcup D\supseteq G\supseteq F$, also
$\bigcup D\in\uparrow F$.

Wende Beweismuster auf $X=\uparrow F$ an mit $\cap$ als $\wedge$.

\begin{definition}
  $I\in\Sat(\lhd)$ {\bf irreduzibel}, wenn aus $I=G\cap H$ folgt $I=G$ oder $I=H$. Das heißt, $I$ ist {\bf reduzibel}, wenn $I=G\cap H$ mit
  $G\supsetneq I$, $H\supsetneq I$ ($G, H\in\Sat(\lhd)$).
\end{definition}

\begin{theorem}[Noether, McCoy, Fuchs, Schmidt]
  Es sei $\lhd$ eine finitäre Überdeckungsrelation auf einer Menge $S$. Für $F\in\Sat(\lhd)$ gilt $F\underbrace{=}_{\mbox{(§)}}\bigcap\{I\in\uparrow
  F\mid I\mbox{ irreduzibel}\}$.
\end{theorem}

\begin{proof}[Beweis mit Beweismuster, also mit OI]
  In (§) ist $\subseteq$ klar. Zu $\supseteq$ sei $a\in\bigcap\{G\in\uparrow F\mid G\mbox{ irreduzibel}\}$. Zeige: $a\in F$, das heißt
  $\forall_{G\in\uparrow F}(a\in G)$. Also ist zu zeigen: $T=X$ mit $T=\{G\in X\mid a\in G\}$ und $X=\uparrow F$. Nun ist $X$ dcpo mit $\cap$; $T$ ist
  $\cap$-abgeschlossen und $T$ offen: Ist $\bigcup D\in T$, das heißt $a\in\bigcup D$, so
  ist $a\in G$ für ein $G\in D$, das heißt $G\in T$ für ein $G\in D$, also $D\between T$ gilt für alle $D\subseteq X$. Nach Voraussetzung an $a$ gilt
  $G\in T$ für alle irreduziblen $G\in X$. Mit Beweismuster ergibt sich $\forall_{G\in X}(G\in T)$, das heißt $T=X$.
\end{proof}

\begin{definition}
  Zusätzlich zur (finitären) Überdeckungsrelation $\lhd$ auf $S$ sei nun gegeben eine {\bf binäre Operation} $\circ:S\times S\rightarrow S$. Dieses
  $\circ$ muss nicht einmal assoziativ sein. $(S,\circ)$ Grouppoid\footnote{{\bf nicht} der kategorientheoretische Groupoid-Begriff}/Magma. Ferner sei
  gegeben ein {\bf ausgezeichnetes Element} $e\in S$ derart, dass für jedes $F\in\Sat(\lhd)$ gilt: aus $e\in F$ folgt $F=S$. Ein $F\in\Sat(\lhd)$
  heißt {\bf eigentlich}, wenn $e\not\in F$.
\end{definition}

\begin{gelaber}
  Betrachte die folgenden beiden Eigenschaften von $\lhd$ und $\circ$:
  \begin{itemize}
    \item {\bf Weakening}. $ab\lhd a$ und $ab\lhd b$ für alle $a,b\in S$ (wobei $x\lhd y \equiv x\lhd\{y\}$, sowie $ab\equiv a\circ b$).
    \item {\bf Weak Right}. $(U,a)^\lhd\cap (U,b)^\lhd\subseteq (U,ab)^\lhd$, das heißt $x\lhd U, a$ und $x\lhd U,b$ $\Rightarrow$ $x\lhd U,ab$ für
      alle $x,a,b\in S$, $U\subseteq S$ (wobei $U,c$ stehe für $U\cup\{c\}$).
  \end{itemize}
\end{gelaber}

\begin{definition}
  Ein Ideal $P$ heißt {\bf Primideal}, wenn aus $ab\in P$ folgt $a\in P$ oder $b\in P$.
\end{definition}

\begin{remark}
  Mit Weakening ist jedes Primideal irreduzibel. [Übung]
\end{remark}

\begin{lemma}
  Mit Weak Right ist jedes irreduzible Ideal ein Primideal.
\end{lemma}

\begin{proof}
  Es sei $I\in\Sat(\lhd)$ irreduzibel, $ab\in I$. Annahme: $a\not\in I\not\ni b$. Nehme $G=(I,a)^\lhd,H=(I,b)^\lhd$. Dann: $G\supsetneq I$;
  $H\supsetneq I$; $G,H\in\Sat(\lhd)$; $I\subseteq G\cap H$; mit Weak Right auch $G\cap H\subseteq (I,ab)^\lhd=I^\lhd=I$.
\end{proof}

\begin{theorem}[Universeller Satz von Krull-Lindenbaum]
  Es sei $\lhd$ eine finitäre Überdeckung und $\circ$ eine binäre Operation auf der Menge $S$ (sowie $e\in S$ ein ausgezeichnetes Element). Gilt Weak
  Right, so ist $F=\bigcap\{P\in\uparrow F\mid P\mbox{ (eigentliches) Primideal}\}$ für jedes $F\in\Sat(\lhd)$.
\end{theorem}

\begin{proof}
  $F\subseteq\bigcap\{P\in\uparrow F\mid P\mbox{ Primideal}\}\underbrace{\subseteq}_{\mbox{\tiny Lemma}}\bigcap\{I\in\uparrow F\mid I\mbox{
    irreduzibel}\}\underbrace{\subseteq}_{\mbox{\tiny Satz N-M-F-S}} F$.
\end{proof}

{\bf Anwendung 1} (Satz von Krull für kommutative Ringe). $R$ kommutativer Ring, $S=R$, $U^\lhd=\sqrt{(U)}$, dann $\lhd$ finitäre
Überdeckungsrelation; ferner sei $\circ$ die Multiplikation in $R$, sowie $e=1$. Weak Right ist hier
$\sqrt{(U,a)}\cap\sqrt{(U,b)}\subseteq\sqrt{(U,ab)}$.

\begin{proof}[Beweis von Weak Right]
  Für $x\in\sqrt{(U,a)}\cap\sqrt{(U,b)}$, etwa $x^k\in(U,a),x^l\in(U,b)$, etwa $x^k=u+ra,x^l=v+tb$ mit $u,v\in U$, ist
  $x^{k+l}=(u+ra)(v+tb)=\underbrace{uv+utb+rav}_{\in(U)}+rtab\in(U,ab)$, also $x\in\sqrt{(U,ab)}$.
\end{proof}

\marginpar{\tiny{18.07.2013}}

Hier sind die (eigentlichen) Ideale bezüglich $\lhd$ genau die (echten) Radikalideale von $R$. Ferner sind die eigentlichen Primideale bezüglich
$\lhd,\circ$ genau die Primideale von $R$. Also gilt $\sqrt{F}=\bigcap\{P\in\Spec(R):P\supseteq F\}$ (Korollar zum Satz von Krull).

%% 18. Juli 2013

{\bf Anwendung 2} (Satz von Artin-Schreier).

\begin{definition}
  $K$ Körper; {\bf Kegel} $P$ in $K$ ist $P\subseteq K$ mit: $x\in K\Rightarrow x^2\in P$; $x,y\in P\Rightarrow x+y, x\cdot y \in P$. Spezielll:
  $P\supseteq Q$; $Q=\{\mbox{Quadratsummen}\}=\{x_1^2+\ldots+x_n^2\}$, sogar: $Q$ kleinster Kegel in $K$.
\end{definition}

\begin{remark}
  Es sei $P$ ein Kegel.
  \begin{itemize}
  \item[a.] Es sei $a\in K^*$. Dann:
    \begin{itemize}
    \item $a\in P \Rightarrow a^{-1}\in P$
    \item $a,-a\in P\Rightarrow -1\in P$
    \end{itemize}
  \item[b.]
    \begin{itemize}
    \item $\Char(K)>0 \Rightarrow -1\in P$
    \item $\Char(K)\neq 2, -1\in P \Rightarrow P=K$
    \end{itemize}
  \end{itemize}
\end{remark}

\begin{proof}
  \begin{itemize}
  \item[a.] $a^{-1}=\underbrace{a}_{\in P}\underbrace{(a^{-1})^2}_{\in Q}\in P$; $-1=\underbrace{-a}_{\in P}\underbrace{a^{-1}}_{\in P}\in P$.
  \item[b.]
    \begin{itemize}
    \item $0 = \underbrace{n}_{\ge 2}\cdot 1=\underbrace{1+\ldots+1}_{n\mbox{-mal}} \Rightarrow -1 = \underbrace{1+\ldots+1}_{\underbrace{(n-1)}_{\ge
        1}-\mbox{-mal}} \in P$.
    \item $x\in K \Rightarrow 4x=\underbrace{\overbrace{(1+x)^2-(1-x)^2}^y}_{\in P\mbox{ falls }-1\in P} \underbrace{\Rightarrow}_{\Char(K)\neq 2}
      x\in P$, wobei $x=\frac{1}{4}y=(2^2)^{-1}y\in P$.
    \end{itemize}
  \end{itemize}
\end{proof}

\begin{definition}
  Kegel $P\left< U\right>$ {\bf erzeugt von} $U\subseteq K$ und Kegel $P$:
  $$ P\left< U\right>=\{a\in K\mid \exists_{n\ge 0}\exists_{u_1,\ldots,u_n\in U}\exists_{(\lambda_v)_{v\in\{0,1\}^n}\in
    P^{\{0,1\}^n}}. a=\sum\limits_{v\in\{0,1\}^n}\lambda_v u^v\} $$
  mit $u^v = u_1^{v_1}\cdot\ldots\cdot u_n^{v_n}$.

  $P\left< U\right>$ kleinster Kegel $\supseteq P\cup U$.
\end{definition}

\begin{theorem}
  Ein $P\subseteq K$ ist ein Kegel genau dann, wenn die Relation $\le$ definiert durch $a\le b :\Leftrightarrow b-a\in P$ eine {\bf Präordnung} oder
  {\bf Quasiordnung} ist (das heißt reflexiv und transitiv), für die gilt: $a\le b\Rightarrow a+c\le b+c$, $a\le b \land c\ge 0\Rightarrow ac\le
  bc$. Mit $\le$ zu $P$ wie oben ist $P=\{x\in K\mid x\ge 0\}$.
\end{theorem}

Nun schreibe $-P = \{-x\mid x\in P\}=\{x\mid -x\in P\}$.

\begin{definition}
  Eine {\bf Ordnung} auf $K$ ist eine Präordnung, die {\bf linear} beziehungsweise {\bf total} ist, das heißt $x\le 0$ oder $x\ge 0$ für $x\in K$
  ({\bf Dichotomie}), und für die gilt, dass sie auch {\bf antisymmetrisch} ist, das heißt $x\le 0 \land x\ge 0\Rightarrow x=0$ für $x\in K$.
\end{definition}

Mit $P$ zu $\le$ wie oben gilt: Antisymmetrie $\Leftrightarrow$ $P\cap -P=\{0\}$ (entscheidend: $\subseteq$), Dichotomie $\Leftrightarrow$ $P\cup -P =
K$ (entscheidend: $\supseteq$).

\begin{remark}
  $P$ Kegel, $\le$ zu $P$ wie oben. Dann
  \begin{itemize}
  \item $\le$ antisymmetrisch $\Leftrightarrow$ $-1\not\in P$
  \item $\le$ dichotom $\Leftrightarrow$ für $x,y\in K$ gilt: $x+y\in P\Rightarrow x\in P \vee y\in P$
  \end{itemize}
\end{remark}

Beweis Übung.

Nun sei $S=K$; $U^\lhd=Q\left<U\right>$ für $U\subseteq S$; $a\circ b = a+b$; $e=-1$. Dann ist $\lhd$ eine finitäre Überdeckungsrelation; die Ideale
sind genau die Kegel, das heißt die Präordnungen; die Primideale sind genau die Ordnungen, das heißt die Kegel $P$ mit $P\cap -P=\{0\}$, $P\cup -P=K$;
die eigentlichen Ideale sind genau die Präordnungen mit $-1\not\in P$.

\begin{lemma}
  $P$ Kegel; $a,b\in K$, $\Char K\neq 2$ $\Rightarrow$ $P\left<a\right>\cap P\left<b\right>\subseteq P\left<a+b\right>$.
\end{lemma}

\begin{proof}
  Es sei $x\in P\left<a\right>\cap P\left<b\right>$, das heißt $x=ya+y'$, $x=zb+z'$ mit $y,y',z,z'\in P$. Dann
  $(y+z)x=yx+zx=y(zb+z')+z(ya+y')=yz(a+b)+yz'+y'z\in P\left<a+b\right>$.

  {\bf Fall 1}. $y+z\in K^*$. Dann $x=\underbrace{(y+z)^{-1}}_{\in P}(y+z)x\in P\left<a+b\right>$.

  {\bf Fall 2}. $y=0$. Dann $x=y'\in P$.

  {\bf Fall 3}. $y+z=0\,\&\,y\in K^*$. Dann $-y=z\in P$, also $y\in P\cap -P$, also $-1\in P$, dann wegen $\Char(K)\neq 2$ folgt $P=K$.
\end{proof}

Der universelle Satz von Krull-Lindenbaum hat hier die folgende Form, wobei $K$ ein Körper mit $\Char(K)\neq 2$ sei:

$$ \bigcap\limits_{\tiny \begin{array}{c}{O\supseteq P} \\ {O\mbox{ Ordnung von } K}\end{array}} O \subseteq P $$

wobei $P$ Präordnung auf $K$.

Spezialfall $P=Q=\{\mbox{Quadratsummen}\}$: Die {\bf total positiven} Elemente von $K$ (das heißt die $x\in K$ mit $x\ge 0$ für jede Ordnung auf $K$)
sind genau die Quadratsummen in $K$.

\begin{theorem}[Artin-Schreier]
  Ein Körper $K$ kann genau dann angeordnet werden (das heißt es gibt eine Ordnung auf $K$), genau dann, wenn $K$ {\bf formal reell} ist (das heißt
  $-1$ ist keine Quadratsumme in $K$).
\end{theorem}

Dieser Satz wurde von Artin verwendet zur Lösung von Hilberts 17. Problem.

{\bf Anwendung 3} (Satz von Lindenbaum). Sei $\cal L$ eine Sprache der Prädikatenlogik erster Stufe. Sei $\cal F$ die Menge der Formeln in $\cal
L$. Sei $\vdash$ die Herleitbarkeit mit klassischer Logik. $\bar{\Gamma} = \{\varphi\in F\mid \Gamma\vdash\varphi\}$ ($\Gamma\subseteq {\cal
  F}$). $\Gamma$ {\bf Theorie} wenn $\Gamma=\bar{\Gamma}$. {\bf Henkin-Erweiterung} ${\cal L}^*$ von $\cal L$: Füge hinzu neue Konstante $c_{\psi,x}$
für jede Formel $\psi\in {\cal F}$ und Variable $x$ mit $\exists_x\psi$ geschlossen. ${\cal F}*$ Formeln von ${\cal L}^*$.

\begin{center}
\begin{tabular}{r||c|c|c}
        & Krull        & Artin-Schreier    & Lindenbaum \\ \hline
$S$     & $R$ Ring     & $K$ Körper        & $\cal F$ \\
$U^\lhd$ & $\sqrt{(U)}$ & $Q\left<U\right>$ & $\bar{U\cup H}$ \\
$\circ$ & $\cdot$      & $+$               & $\vee$ \\
$e$     & $1$          & $-1$              & $\bot$
\end{tabular}
\end{center}

\begin{definition}
  {\bf Henkin-Menge} $\Theta$: Theorie $\Theta\subseteq{\cal L}^*$ mit $\bot\not\in\Theta$ und
  \begin{itemize}
    \item[i.] $\psi_1\vee\psi_2\in\Theta\Rightarrow \psi_1\in\Theta \vee \psi_2\in\Theta$
    \item[ii.] $(\exists_{x}\psi)\in\Theta$, $(\exists_x\psi)\mbox{ geschlossen}$ $\Rightarrow$ es gibt Term $t$ mit $\psi[x/t]\in\Theta$.
  \end{itemize}
\end{definition}

$H$: Menge der {\bf Henkin-Axiome} der Form $\exists x \psi\rightarrow\psi[x/c_{\psi,x}]$ ($\psi$,$x$,$c_{\psi,x}$ wie oben).

Falls $\Theta\subseteq{\cal L}^*$ Theorie mit $\Theta\supseteq H$, so ist (ii.) automatisch erfüllt (modus ponens).

Es sei $S={\cal F}^*$, sowie $\Gamma^\lhd=\bar{\Gamma\cup H}$ für $\Gamma\subseteq{\cal F}^*$. Dann: Die Ideale sind die Theorien
$\Theta\subseteq{\cal L}^*$ mit $\Theta\supseteq H$; die eigentlichen Primideale sind die Henkin-Mengen, wobei $\circ$ stehe für $\vee$ und $e$ für
$\bot$. Dabei ist $e=\bot$ ausgezeichnet: $\bot\in\Theta$, $\Theta$ Ideal $\Rightarrow$ $\Theta={\cal F}^*$ wegen ex falso quodlibet:
$\bot\rightarrow\varphi$.

Weak Right:

$\varphi\lhd\Gamma,\psi_1\wedge\varphi\lhd\Gamma,\psi_2\Rightarrow\varphi\lhd\Gamma,\psi_1\circ\psi_2$

$\Gamma,H,\psi_1\vdash\varphi \wedge \Gamma,H,\psi_2\vdash\varphi\Rightarrow\Gamma,H,\psi_1\vee\psi_2\vdash\varphi$

Folgt aus: $\Gamma,\psi_1\vdash\varphi \wedge \Gamma,\psi_2\vdash\varphi\Rightarrow\Delta,\psi_1\vee\psi_2\vdash\varphi$, das heißt aus der
Eliminationsregel für die Disjunktion. Weakening folgt aus der Einführungsregel für Disjunktion. $\lhd$ ist eine finitäre Überdeckungsrelation. Hier
lautet der universelle Satz von Krull-Lindenbaum:

$$ \bigcap\limits_{\tiny \begin{array}{c}\Theta\mbox{ Henkin-Menge}\\ \Theta\supseteq\Gamma\cup H\end{array}}\Theta \subseteq \bar{\Gamma\cup H}
\qquad (\Gamma\subseteq{\cal F}^*) $$

für $\varphi\in{\cal F}^*$, $\Gamma\subseteq{\cal F}^*$ gilt: ist $\varphi\in\Theta$ für jede Henkin-Menge $\Theta$ mit $\Theta\supseteq\Gamma\cup H$,
so gilt $\Gamma,H\vdash\varphi$.

{\bf Fakten.}
\begin{itemize}
\item[a.] $\Gamma,H\vdash\varphi\Rightarrow\Gamma\vdash\varphi$, wobei $\Gamma\subseteq{\cal F}$ und $\varphi\in{\cal F}$.
\item[b.] Für jedes $\Theta\subseteq{\cal F}^*$ mit $\Theta\supseteq H$ sind äquivalent:
  \begin{itemize}
  \item[i.] $\Theta$ ist eine Henkin-Menge (das heißt $\Theta$ Theorie, $\bot\not\in\Theta$,
    $\psi_1\vee\psi_2\in\Theta\Rightarrow\psi_1\in\Theta\vee\psi_2\in\Theta$.
  \item[ii.] $\Theta$ ist eine vollständige ($\psi\in\Theta\vee(\lnot\psi)\in\Theta$), konsistente Theorie
    ($\bot\not\in\Theta\,\&\,\Theta=\bar{\Theta}$).
  \item[iii.] $\Theta$ ist eine maximale konsistente Theorie.
  \end{itemize}
\end{itemize}

Beweis Übung. Beachte: $\top\in\Theta$, falls $\Theta$ Theorie; $\top$ ``ist'' $\varphi\vee\lnot\varphi$.

Damit folgt der Gödel'sche Vollständigkeitssatz!

{\bf Fallstudie.} Von idealen Objekten zu endlichen Beweismethoden.

Sei $R$ kommutativer Ring mit $1$; $f,g\in R[T]$, $f=\sum\limits_{i=0}^n a_iT^i, g=\sum\limits_{j=0}^m b_jT^j$. Es sei $i_0\in\{1,\ldots,n\}$ beliebig
aber fest; setze $u=a_{i_0}$.

Betrachte NC: $$fg=1\Rightarrow\underbrace{\exists_k.u^k=0}_{\in\sqrt{0}}$$.

Mit $fg=\sum c_kT^k$, $c_k=\sum\limits_{i+j=k}a_jb_j$ lautet $fg=1$ wie folgt:

$$ c_0=1, c_1=0,\ldots, c_{n+m} = 0  $$

das heißt

$$ a_0 b_0 = 1, a_0 b_1 + a_1 b_0 = 0, \ldots, a_n b_m = 0 $$

{\bf Kurzer und eleganter Beweis} von NC mit Satz von Krull, also mit ZL/OI:

{\bf Spezialfall} $R$ Integritätsring: $\underbrace{\deg(fg)}_0=\underbrace{\deg(f)}_0+\underbrace{\deg(g)}_0$. Also sogar $u=0$.

{\bf Allgemeiner Fall} $R$ beliebig: Reduzierung auf Spezialfall: Gehe von $R$ nach $R/P$ für jedes Primideal $P$. Dann ist $R/P$
Integritätsring. Dort ist $u=0$ in $R/P$, das heißt $u=0 \mod P$, das heißt $u\in P$.

Also gesehen: $fg=1\Rightarrow\forall_P(u\in P)$, das heißt $u\in\bigcap\Spec(R)$.

Wollen zeigen: $fg=1\Rightarrow\exists_k(u^k=0)$, das heißt $u\in\sqrt{0}$.

Mit Satz von Krull ($\bigcap\Spec(R)=\sqrt{0}$) fertig.

{\bf Alternativer Beweis} von NC mit vollständiger Induktion.

{\bf Finite Induction} (FI): Ist $X$ endliche partiell geordnete Menge und $U\subseteq X$ progressiv, so ist $U=X$.

Folgt mit vollständiger Induktion nach $|X|$. Zeige zunächst: $X$ hat ein maximales Element $x$. Dann: $\underbrace{U}_{\subseteq X}$ progressiv,
$\underbrace{x}_{\in X}$ maximal $\Rightarrow$ $x\in U$.

Für NC bestehe $X$ aus allen Idealen der Form $(D)$ mit $D\subseteq\{a_1,\ldots,a_n\}\cup\{b_1,\ldots,b_m\}$. Dieses $X$ ist endlich, partiell
geordnet durch $\subseteq$. Es sei $U=\{F\in X\mid u\in\sqrt{F}\}$. Zu zeigen: $u\in\sqrt{0}$, das heißt $\forall_{F\in X}\underbrace{(F\in
  U)}_{U\in\sqrt{F}}$, das heißt $U=X$. Mit FI zu zeigen: $U$ progressiv, das heißt für $F\in X$ folgt $F\in U$ aus $\forall_{G\in X}.G\supsetneq
F\Rightarrow G\in U$. Sei also $F\in X$ so. Zeige: $F\in U$.
\begin{itemize}
\item Fall 1: $\forall_{i>0}.a_i\in F$. Dann $u\in F$, also $F\in U$.
\item Fall 2: $\exists_{i>0}.a_i\not\in F$. Dann auch $\exists_{j>0}.b_j\not\in F$. Wähle $i, j$ maximal. Dann $F\ni
  0=\underbrace{\sum\limits_{\tiny \begin{array}{c}p+q = i+j\\ q>j\end{array}} a_pb_q}_{\in F} +
  a_jb_j+\underbrace{\sum\limits_{\tiny \begin{array}{c}p+q=i+j\\ p>i\end{array}} a_pb_q}_{\in F}\Rightarrow a_ib_j\in
  F$. $G=F+(a_i),H=F+(b_j)\Rightarrow G,H\in X$; $G,H\supsetneq F$ $\underbrace{\Rightarrow}_{\mbox{Ind}}$ $G,H\in U$, das heißt $u^k\in G, u^l\in H$,
  also $u^{k+l}\in G\cap H$, das heißt $u\in\sqrt{G}\cap\sqrt{H}\underbrace{\subseteq}_{\mbox{Weak Right}}\sqrt{F+(a_jb_j)}\underbrace{=}_{a_ib_j\in
    F}\sqrt{F}$, also $F\in U$.
\end{itemize}

\end{document}
